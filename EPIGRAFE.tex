\pagebreak\paginabranca
\thispagestyle{empty}

\begin{Parallel}[p]{}{}
\ParallelRText{


\vspace*{\fill}

\epigraph{}{}{
\begin{flushright}
A Léon Werth.\\
\end{flushright}
Peço perdão às crianças por ter dedicado este livro a uma pessoa grande.
Tenho uma desculpa muito séria: essa pessoa grande foi o melhor amigo
que tive no mundo. Tenho uma outra desculpa: essa pessoa grande é capaz
de entender tudo, até os livros para as crianças. Tenho uma terceira
desculpa: essa pessoa grande mora na França e sente fome e frio. Ela
precisa ser consolada. Se todas essas desculpas não bastarem, eu
gostaria de dedicar esse livro à criança que essa pessoa grande foi um
dia. Todas as pessoas grandes foram crianças. (Ainda que poucas se
lembrem disso.) Então vou corrigir minha dedicatória:
\begin{flushright}
A Léon Werth,

quando ele ainda era um menino.
\end{flushright}
}}

\ParallelLText{


\vspace*{\fill}

\epigraph{}{}{
\begin{flushright}
À Léon Werth.\\
\end{flushright}
Je demande pardon aux enfants d'avoir
dédié ce livre à une grande personne. J'ai
une excuse sérieuse: cette grande personne est le meilleur ami que j'ai au
monde. J'ai une autre excuse: cette grande
personne peut tout comprendre, même les
livres pour enfants. J'ai une troisième
excuse: cette grande personne habite la
France où elle a faim et froid. Elle a bien
besoin d'être consolée. Si toutes ces excuses ne suffisent pas, je veux bien dédier ce livre à l'enfant qu'a été autrefois cette grande personne. Toutes les grandes
personnes ont d'abord été des enfants.
(Mais peu d'entre elles s'en souviennent.)
Je corrige donc ma dédicace:
\begin{flushright}
À Léon Werth,

quand il était petit garçon.
\end{flushright}
}}
\end{Parallel}
