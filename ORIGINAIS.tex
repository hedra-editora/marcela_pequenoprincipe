%\textbf{\emph{O pequeno príncipe}}

%\textbf{Antoine de Saint-Exupéry}
%\afterpage{\blankpage}

\pagebreak\paginabranca
\thispagestyle{empty}

\begin{Parallel}[p]{}{}
\ParallelRText{


\vspace*{\fill}

\epigraph{}{}{
\begin{flushright}
A Léon Werth.\\
\end{flushright}
Peço perdão às crianças por ter dedicado este livro a uma pessoa grande.
Tenho uma desculpa muito séria: essa pessoa grande foi o melhor amigo
que tive no mundo. Tenho uma outra desculpa: essa pessoa grande é capaz
de entender tudo, até os livros para as crianças. Tenho uma terceira
desculpa: essa pessoa grande mora na França e sente fome e frio. Ela
precisa ser consolada. Se todas essas desculpas não bastarem, eu
gostaria de dedicar esse livro à criança que essa pessoa grande foi um
dia. Todas as pessoas grandes foram crianças. (Ainda que poucas se
lembrem disso.) Então vou corrigir minha dedicatória:
\begin{flushright}
A Léon Werth,

quando ele ainda era um menino.
\end{flushright}
}}

\ParallelLText{


\vspace*{\fill}

\epigraph{}{}{
\begin{flushright}
À Léon Werth.\\
\end{flushright}
Je demande pardon aux enfants d'avoir
dédié ce livre à une grande personne. J'ai
une excuse sérieuse: cette grande personne est le meilleur ami que j'ai au
monde. J'ai une autre excuse: cette grande
personne peut tout comprendre, même les
livres pour enfants. J'ai une troisième
excuse: cette grande personne habite la
France où elle a faim et froid. Elle a bien
besoin d'être consolée. Si toutes ces excuses ne suffisent pas, je veux bien dédier ce livre à l'enfant qu'a été autrefois cette grande personne. Toutes les grandes
personnes ont d'abord été des enfants.
(Mais peu d'entre elles s'en souviennent.)
Je corrige donc ma dédicace:
\begin{flushright}
À Léon Werth,

quand il était petit garçon.
\end{flushright}
}}
\end{Parallel}




\begin{Parallel}[p]{}{}
\ParallelLText{
\textbf{I}

Lorsque j'avais six ans j'ai vu, une fois, une magnifique image, dans un livre sur la forêt vierge qui s'appelait \emph{Histoires vécues}. Ça représentait un serpent boa qui avalait un fauve. Voilà la copie du dessin.

On disait dans le livre: ``Les serpents boas avalent
leur proie tout entière, sans la mâcher. Ensuite ils ne
peuvent plus bouger et ils dorment pendant les six
mois de leur digestion''.

J'ai alors beaucoup réfléchi sur les aventures de la
jungle et, à mon tour, j'ai réussi, avec un crayon de
couleur, à tracer mon premier dessin. Mon dessin
numéro 1.~Il~était~comme~ça:

J'ai montré mon chef-d'œuvre aux grandes personnes et je leur ai demandé si mon dessin leur faisait peur.

Elles m'ont répondu: ``Pourquoi un chapeau
ferait-il peur?''

Mon dessin ne représentait pas un chapeau. Il représentait un serpent boa qui digérait un éléphant. J'ai alors dessiné l'intérieur du serpent boa, afin que les grandes personnes puissent comprendre. Elles ont toujours besoin d'explications.
Mon dessin numéro 2 était comme~ça:

Les grandes personnes m'ont conseillé de laisser
de côté les dessins de serpents boas ouverts ou fermés, et de m'intéresser plutôt à la géographie, à
l'histoire, au calcul et à la grammaire. C'est ainsi
que j'ai abandonné, à l'âge de six ans, une magnifique carrière de peintre. J'avais été découragé par
l'insuccès de mon dessin numéro 1 et de mon dessin numéro 2. Les grandes personnes ne comprennent jamais rien toutes seules, et c'est fatigant, pour
les enfants, de toujours et toujours leur donner des
explications\ldots{}

J'ai donc dû choisir un autre métier et j'ai appris à
piloter des avions. J'ai volé un peu partout dans le
monde. Et la géographie, c'est exact, m'a beaucoup
servi. Je savais reconnaître, du premier coup d'œil,
la Chine de l'Arizona. C'est très utile, si l'on s'est
égaré pendant la nuit.

J'ai ainsi eu, au cours de ma vie, des tas de
contacts avec des tas de gens sérieux. J'ai beaucoup
vécu chez les grandes personnes. Je les ai vues de
très près. Ça n'a pas trop amélioré mon opinion.

Quand j'en rencontrais une qui me paraissait un
peu lucide, je faisais l'expérience sur elle de mon
dessin numéro 1 que j'ai toujours conservé. Je voulais savoir si elle était vraiment compréhensive. Mais
toujours elle me répondait: ``C'est un chapeau''.
Alors je ne lui parlais ni de serpents boas, ni de forêts vierges, ni d'étoiles. Je me mettais à sa portée.
Je lui parlais de bridge, de golf, de politique et de
cravates. Et la grande personne était bien contente
de connaître un homme aussi raisonnable\ldots{}\\

\textbf{II}

J'ai ainsi vécu seul, sans personne avec qui parler
véritablement, jusqu’à une panne dans le désert du
Sahara, il six ans. Quelque chose s'était cassé
dans mon moteur. Et comme je n'avais avec moi ni
mécanicien, ni passagers, je me préparai à essayer
de réussir, tout seul, une réparation difficile. C'était
pour moi une question de vie ou de mort. J'avais à
peine de l'eau boire pour huit jours.

Le premier soir je me suis donc endormi sur le
sable à mille milles de toute terre habitée. J'étais
bien plus isolé qu'un naufragé sur un radeau au
milieu de l'océan. Alors vous imaginez ma surprise,
au lever du jour, quand une drôle de petite voix m'a
réveillé. Elle disait:

-- S'il vous plaît\ldots{} dessine-moi un mouton!

-- Hein!

-- Dessine-moi un mounton\ldots{}

J'ai sauté sur mes pieds comme si j'avais été
frappé par la foudre. J'ai bien frotté mes yeux. J'ai bien regardé. Et j'ai vu un petit bonhomme tout à
fait extraordinaire qui me considérait gravement.
Voilà le meilleur portrait que, plus tard, j'ai réussi à
faire de lui. Mais mon dessin, bien sûr, est beaucoup
moins ravissant que le modèle. Ce n'est pas ma
faute. J'avais été découragé dans ma carrière de
peintre par les grandes personnes, à l'âge de six ans,
et je n'avais rien appris à dessiner, sauf les boas fermés et les boas ouverts.

Je regardai donc cette apparition avec des yeux
tout ronds d'étonnement. N'oubliez pas que je me
trouvais à mille milles de toute région habitée. Or
mon petit bonhomme ne me semblait ni égaré, ni
mort de fatigue, ni mort de faim, ni mort de soif, ni
mort de peur. Il n'avait en rien l'apparence d’un
enfant perdu au milieu du dé sert, à mille milles de
toute région habitée. Quand je réussis enfin à parler, je luis dis:

-- Mais\ldots{} qu'est-ce que tu fais là?

Et il me répéta alors, tout doucement, comme
une chose très sérieuse:

-- S'il vous plaît\ldots{} dessine-moi un mouton\ldots{}

Quand le mystère est trop impressionnant, on
n'ose pas désobéir. Aussi absurde que cela me semblât à mille milles de tous les endroits habités et en
danger de mort, je sortis de ma poche une feuille de
papier et un stylographe. Mais je me rappelai alors
que j'avais surtout étudié la géographie, l'histoire, le
calcul et la grammaire et je dis au petit bonhomme
(avec un peu de mauvaise humeur) que je ne savais
pas dessiner. Il me répondit:

-- Ça ne fait rien. Dessine-moi un mouton.

Comme je n'avais jamais dessiné un mouton je relis, pour lui, l'un des deux seuls dessins dont j'étais capable. Celui du boa fermé. Et je fus stupéfait d'entendre le
petit bonhomme me répondre:

-- Non! Non! Je ne veux pas d'un
éléphant dans un boa. Un boa c'est
très dangereux, et un éléphant c'est très encombrant. Chez moi c'est tout petit. J'ai besoin d'un
mouton. Dessine-moi un mouton.

Alors j'ai dessiné.

Il regarda attentivement, puis:

-- Non! Celui-là est très malade. Fais-en un autre.

Je dessinai.

Mon ami sourit gentiment, avec
indulgence:

-- Tu vois bien\ldots{} ce n'est pas un mouton, c'est un
bélier. Il a des cornes\ldots{}

Je refis donc encore mon dessin.

Mais il fut refusé, comme les précédents:

-- Celui-là est trop vieux. Je veux un
mouton qui vive longtemps.

Alors, faute de patience, comme
j'avais hâte de commencer le démontage de mon moteur, je griffonnai ce
dessin-ci.

Et je lançai:

-- Ça c'est la caisse. Le mouton que tu
veux est dedans.

Mais je fus bien surpris de voir s'illuminer le
visage de mon jeune juge:

-- C'est tout à fait comme ça que je le voulais!
Crois-tu qu'il faille beaucoup d'herbe à ce mouton?

-- Pourquoi?

-- Parce que chez moi c'est tout petit\ldots{}

-- Ça suffira sûrement. Je t'ai donné un tout
petit mouton.

Il pencha la tête vers le dessin:

-- Pas si petit que ça\ldots{} Tiens! Il s'est endormi\ldots{}

Et c'est ainsi que je fis la connaissance du petit
prince.\\

\textbf{III}

Il me fallut longtemps pour comprendre d'où il
venait. Le petit prince, qui me posait beaucoup de
questions, ne semblait jamais entendre les miennes.
Ce sont des mots prononcés par hasard qui, peu à
peu, m'ont tout révélé. Ainsi, quand il aperçut
pour la première fois mon avion (je ne dessinerai
pas mon avion, c'est un dessin beaucoup trop
compliqué pour moi) il me demanda:

-- Qu'est-ce que c'est que cette chose-là?

-- Ce n'est pas une chose. Ça vole. C'est un avion. C'est mon avion.
 
Et j’étais fier de lui apprendre que je volais. Alors il
s'écria:

-- Comment! tu es tombé du ciel!

-- Oui, fis-je modestement.

-- Ah! ça c'est drôle!\ldots{}

Et le petit prince eut un très joli éclat de rire qui
m'irrita beaucoup. Je désire que l'on prenne mes
malheurs au sérieux. Puis il ajouta:

-- Alors, toi aussi tu viens du ciel! De quelle planète es-tu?

J'entrevis aussitôt une lueur, dans le mystère de sa
présence, et j'interrogeai brusquement:

-- Tu viens donc d'une autre planète?

Mais il ne me répondit pas. Il hochait la tête doucement tout en regardant mon avion:

-- C'est vrai que, là-dessus, tu ne peux pas venir de
bien loin\ldots{}

Et il s'enfonça dans une rêverie qui dura longtemps. Puis, sortant mon mouton de sa poche, il se
plongea dans la contemplation de son trésor.

\medskip

Vous imaginez combien j'avais pu être intrigué
par cette demi-confidence sur ``les autres planètes''.
Je m'efforçai donc d'en savoir plus long:

-- D'où viens-tu, mon petit bonhomme? Où est-ce
``chez toi''? Où veux-tu emporter mon mouton?

Il me répondit après un silence méditatif:

-- Ce qui est bien, avec la caisse que tu m'as donnée, c'est que, la nuit, ça lui servira de maison.

-- Bien sûr. Et si tu es gentil, je te donnerai aussi
une corde pour l'attacher pendant le jour. Et un
piquet.

La proposition parut choquer le petit prince:

-- L'attacher? Quelle drôle d'idée!

-- Mais si tu ne l'attaches pas, il ira n'importe où,
et il se perdra.

Et mon ami eut un nouvel éclat de rire:

-- Mais où veux-tu qu'il aille!

-- N'importe où. Droit devant lui\ldots{}

Alors le petit prince remarqua gravement:

-- Ça ne fait rien, c'est tellement petit, chez moi!

Et, avec un peu de mélancolie, peut-être, il
ajouta:

-- Droit devant soi on ne peut pas aller bien loin\ldots{}\\

\textbf{IV}

J'avais ainsi appris une seconde chose très importante: c'est que sa planète d'origine était à peine
plus grande qu'une maison!

Ça ne pouvait pas m'étonner beaucoup. Je savais
bien qu'en dehors des grosses planètes comme la
Terre, Jupiter, Mars, Vénus, auxquelles on a donné des
noms, il y en a des centaines d'autres
qui sont quelquefois si petites qu'on a
beaucoup de mal à les apercevoir au télescope. Quand
un astronome découvre l'une d'elles, il lui donne pour
nom un numéro. Il l'appelle par exemple: ``l'astéroïde 325''.

J'ai de sérieuses raisons de croire que la planète d'où venait le petit prince est l'astéroïde B 612.
Cet astéroïdc n'a été aperçu qu'une fois au télescope, en 1909, par un astronome turc.

Il avait fait alors une grande
démonstration de sa découverte à un
congrès international d'astronomie. Mais personne
ne l'avait cru à cause de son costume. Les grandes
personnes sont comme ça.

Heureusement pour la réputation de l'astéroïde B 612, un dictateur turc imposa à son peuple, sous
peine de mort, de s'habiller à l'européenne. L'astronome refit sa démonstration en 1920, dans un habit
très élégant. Et cette fois-ci tout le monde fut de son avis.

Si je vous ai raconté ces détails sur l'astéroïde B 612 et si je vous ai confié son numéro, c'est
à cause des grandes personnes. Les grandes personnes aiment les chiffres. Quand vous leur parlez
d'un nouvel ami, elles ne vous questionnent jamais
sur l'essentiel. Elles ne vous disent jamais: ``Quel
est le son de sa voix? Quels sont les jeux qu'il préfère? Est-ce qu'il
collectionne les papillons?'' Elles vous demandent: ``Quel âge a-t-il? Combien a-t-il de frères? Combien pèse-t-il? Combien gagne son père?'' Alors seulement elles croient le connaître. Si
vous dites aux grandes personnes: ``J'ai vu une belle
maison en briques roses, avec des géraniums aux
fenêtres et des colombes sur le toit\ldots{}'', elles ne parviennent pas à s'imaginer cette maison. Il faut leur
dire: ``J'ai vu une maison de cent mille francs''.
Alors elles s'écrient: ``Comme c'est joli!''

Ainsi, si vous leur dites, La preuve que le petit
prince a existé c'est qu'il était ravissant, qu'il riait, et
qu'il voulait un mouton. Quand on veut un mouton,
c'est la preuve qu'on existe'', elles hausseront les
épaules et vous traiteront d'enfant! Mais si vous leur
dites: ``La planète d'où il venait est l'astéroïde B 
612'', alors elles seront convaincues, et elles vous
laisseront tranquille avec leurs questions. Elles sont
comme ça. Il ne faut pas leur en vouloir. Les enfants
doivent être très indulgents envers les grandes personnes.

Mais, bien sûr, nous qui comprenons la vie, nous
nous moquons bien des numéros! J'aurais aimé
commencer cette histoire à la façon des contes de
fées. J'aurais aimé dire:

``Il était une fois un petit prince qui habitait une
planète à peine plus grande que lui, et qui avait
besoin d'un ami\ldots{}'' Pour ceux qui comprennent la
vie, ça aurait eu l'air beaucoup plus vrai.

Car je n'aime pas qu'on lise mon livre à la légère.
J'éprouve tant de chagrin à raconter ces souvenirs. Il y a
six ans déjà que mon ami s'en est allé avec son
mouton. Si j'essaie ici de le décrire, c'est afin de ne
pas l'oublier. C'est triste d'oublier un ami. Tout le
monde n'a pas eu un ami. Et je puis devenir comme les grandes personnes qui ne s'intéressent plus
qu'aux chiffres. C'est donc pour ça encore que j'ai
acheté une boîte de couleurs et des crayons. C'est
dur de se remettre au dessin, à mon âge, quand on
n'a jamais fait d'autres tentatives que celle d'un boa
fermé et celle d'un boa ouvert, à l'âge de six ans!
J'essaierai, bien sûr, de faire des portraits le plus ressemblants possible. Mais je ne suis pas tout à fait certain de réussir. Un dessin va, et l'autre ne ressemble
plus. Je me trompe un peu aussi sur la taille. Ici le
petit prince est trop grand. Là il est trop petit. J'hésite aussi sur la couleur de son costume. Alors je
tâtonne comme ci et comme ça, tant bien que mal.
Je me tromperai enfin sur certains détails plus
importants. Mais ça, il faudra me le pardonner. Mon
ami ne donnait jamais d'explications. Il me croyait
peut-être semblable à lui. Mais moi, malheureusement, je ne sais pas voir les moutons à travers les
caisses. Je suis peut-être un peu comme les grandes
personnes. J'ai dû vieillir.\\

\textbf{V}

Chaque jour j'apprenais quelque chose sur la
planète, sur le départ, sur le voyage. Ça venait tout
doucement, au hasard des réflexions. C'est ainsi
que, le troisième jour, je connus le drame des baobabs.

Cette fois-ci encore ce fut grâce au mouton, car
brusquement le petit prince m'interrogea, comme
pris d'un doute grave:

-- C'est bien vrai, n'est-ce pas, que les moutons
mangent les arbustes?

-- Oui. C'est vrai.

-- Ah! Je suis content!

Je ne compris pas pourquoi il était si important
que les moutons mangeassent les arbustes. Mais le
petit prince ajouta:

-- Par conséquent ils mangent aussi les baobabs?

Je fis remarquer au petit prince que les baobabs
ne sont pas des arbustes, mais des arbres grands
comme des églises et que, si même il emportait avec
lui tout un troupeau d'éléphants, ce troupeau ne
viendrait pas à bout d'un seul baobab.

L'idée du troupeau d'éléphants fit rire le petit
prince:

-- Il faudrait les mettre les uns sur les autres\ldots{}

Mais il remarqua avec sagesse:

-- Les baobabs, avant de grandir, ça commence par
être petit.

-- C'est exact! Mais pourquoi veux-tu que tes
moutons mangent les petits baobabs?

Il me répondit:

-- Ben! Voyons!, comme s'il s'agissait là d'une évidence.

Et il me fallut un grand effort d'intelligence pour comprendre à moi seul ce problème.

Et en effet, sur la planète du petit prince, il y
avait, comme sur toutes les planètes, de bonnes herbes et de mauvaises herbes. Par conséquent de bonnes graines de bonnes herbes et de mauvaises graines de mauvaises herbes. Mais les graines sont
invisibles. Elles dorment dans le secret de la terre
jusqu'à ce qu'il prenne fantaisie à l'une d'elles de se
réveiller. Alors elle s'étire, et pousse d'abord timidement vers le soleil une ravissante petite brindille
inoffensive. S'il s'agit d'une brindille de radis ou de
rosier, on peut la laisser pousser comme elle veut.
Mais s'il s'agit d'une mauvaise plante, il faut arracher la plante aussitôt, dès qu'on a su la reconnaître. Or il y avait des graines terribles sur la planète du petit prince\ldots{} c'étaient les graines de
baobabs. Le sol de la planète en était infesté. Or un
baobab, si l'on s'y prend trop tard, on ne peut
jamais plus s'en débarrasser. Il encombre toute la
planète. Il la perfore de ses racines. Et si la planète
est trop petite, et si les baobabs sont trop nombreux,
ils la font éclater.

-- C'est une question de discipline, me disait plus
tard le petit prince. Quand on a terminé sa toilette
du matin, il faut faire soigneusement la toilette de la
planète. Il faut s'astreindre régulièrement à arracher les baobabs dès qu'on les distingue d'avec les
rosiers auxquels ils ressemblent beaucoup quand ils
sont très jeunes. C'est un travail très ennuyeux, mais
très facile.

Et un jour il me conseilla de m'appliquer à réussir
un beau dessin, pour bien faire entrer ça dans la tête
des enfants de chez moi.

-- S'ils voyagent un jour, me disait-il, ça pourra leur servir. Il est quelquefois sans
inconvénient de remettre à plus tard son travail.
Mais, s'il s'agit des baobabs, c'est toujours une catastrophe. J'ai connu une planète, habitée par un paresseux . Il avait négligé trois arbustes\ldots{}

Et, sur les indications du petit prince, j'ai dessiné
cette planète-là. Je n'aime guère prendre le ton
d'un moraliste. Mais le danger des baobabs est si peu connu, et les risques courus par celui qui s'égarerait dans un astéroïde sont si considérables, que, pour une fois, je fais exception à ma réserve. Je dis:

-- Enfants! Faites attention aux baobabs!

C'est pour avertir mes amis d'un danger qu'ils frôlaient
depuis longtemps, comme moi-même, sans le
connaître, que j'ai tant travaillé ce dessin-là. La
leçon que je donnais en valait la peine. Vous vous
demanderez peut-être: Pourquoi n'y a-t-il pas, dans
ce livre, d'autres dessins aussi grandioses que le dessin des baobabs? La réponse est bien simple: J'ai
essayé mais je n'ai pas pu réussir. Quand j'ai dessiné les baobabs j'ai été animé par le sentiment de l'urgence.\\

\textbf{VI}

Ah! petit prince, j'ai compris, peu à peu, ainsi, ta
petite vie mélancolique. Tu n'avais eu longtemps
pour distraction que la douceur des couchers de
soleil. J'ai appris ce détail nouveau, le quatrième
jour au matin, quand tu m'as dit:

-- J'aime bien les couchers de soleil. Allons voir un
coucher de soleil\ldots{}

-- Mais il faut attendre\ldots{}

-- Attendre quoi?

-- Attendre que le soleil se couche.

Tu as eu l'air très surpris d'abord, et puis tu as ri
de toi-même. Et tu m'as dit:

-- Je me crois toujours chez moi!

En effet. Quand il est midi aux États-Unis, le
soleil, tout le monde le sait, se couche sur la France.
Il suffirait de pouvoir aller en France en une minute
pour assister au coucher du soleil. Malheureusement la France est bien trop éloignée. Mais, sur ta si
petite planète, il te suffisait de tirer ta chaise de
quelques pas. Et tu regardais le crépuscule chaque
fois que tu le désirais\ldots{}

-- Un jour, j'ai vu le soleil se coucher quarante-quatre fois!

Et un peu plus tard tu ajoutais:

-- Tu sais\ldots{} quand on est tellement triste on aime
les couchers de soleil\ldots{}

-- Le jour des quarante-quatre fois, tu étais donc
tellement triste?

Mais le petit prince ne répondit pas.\\

\textbf{VII}

Le cinquième jour, toujours grâce au mouton, ce
secret de la vie du petit prince me fut révélé. Il
me demanda avec brusquerie, sans préambule,
comme le fruit d'un problème longtemps médite en
silence:

-- Un mouton, s'il mange les arbustes, il mange
aussi les fleurs?

-- Un mouton mange tout ce qu'il rencontre.

-- Même les fleurs qui ont des épines?

-- Oui. Même les fleurs qui ont des épines.

-- Alors les épines, à quoi servent-elles?

Je ne le savais pas. J'étais alors très occupé à essayer de dévisser un boulon trop serré de mon
moteur. J'étais très soucieux car ma panne commençait de m'apparaître comme très grave, et l'eau à
boire qui s'épuisait me faisait craindre le pire.

-- Les épines, à quoi servent-elles?

Le petit prince ne renonçait jamais à une question, une fois qu'il l'avait posée. J'étais irrité par
mon boulon et je répondis n'importe quoi:

-- Les épines, ça ne sert à rien, c'est de la pure méchanceté de la part des fleurs!

-- Oh!

Mais après un silence il me lança, avec une sorte
de rancune:

-- Je ne te crois pas! Les fleurs sont faibles. Elles
sont naïves. Elles se rassurent comme elles peuvent.
Elles se croient terribles avec leurs épines\ldots{}

Je ne répondis rien. À cet instant-là je me disais: ``Si ce boulon résiste encore, je le ferai sauter d'un
coup de marteau''. Le petit prince dérangea de nouveau mes réflexions:

-- Et tu crois, toi, que les fleurs\ldots{}

-- Mais non! Mais non! Je ne crois rien! J'ai
répondu n'importe quoi. Je m'occupe, moi, de
choses sérieuses!

Il me regarda stupéfait.

-- De choses sérieuses!

Il me voyait, mon marteau à la main, et les doigts
noirs de cambouis, penché sur un objet qui lui semblait très laid.

-- Tu parles comme les grandes personnes!

Ça me fit un peu honte. Mais, impitoyable, il ajouta:

-- Tu confonds tout\ldots{} tu mélanges tout!

Il était vraiment très irrité. Il secouait au vent des
cheveux tout dorés:

-- Je connais une planète où il y a un monsieur cramoisi. Il n'a jamais respiré une fleur. Il n'a jamais
regardé une étoile. Il n'a jamais aimé personne. Il
n'a jamais rien fait d'autre que des additions. Et
toute la journée il répète comme toi: ``Je suis un
homme sérieux! Je suis un homme sérieux!'', et ça
le fait gonfler d'orgueil. Mais ce n'est pas un
homme, c'est un champignon!

-- Un quoi?

-- Un champignon!

Le petit prince était maintenant tout pâle de colère.

-- Il y a des millions d'années que les fleurs fabriquent des
épines. Il y a des millions d'années que les moutons mangent
quand même les fleurs. Et ce n'est pas sérieux de chercher à
comprendre pourquoi elles se
donnent tant de mal pour se
fabriquer des épines qui ne servent jamais à rien? Ce n'est pas
important la guerre des moutons et des fleurs? Ce n'est pas
plus sérieux et plus important que les additions d'un gros
monsieur rouge? Et si je connais, moi, une fleur unique
au monde, qui n'existe nulle part, sauf dans ma planète,
et qu'un petit mouton peut anéantir d'un seul coup, comme ça, un matin, sans
se rendre compte de ce qu'il fait, ce n'est pas important ça!

Il rougit, puis reprit:

-- Si quelqu'un aime une fleur qui n'existe qu'à un
exemplaire dans les millions et les millions d'étoiles,
ça suffit pour qu'il soit heureux quand il les
regarde. Il se dit: ``Ma fleur est là quelque part\ldots{}''
Mais, si le mouton mange la fleur, c'est pour lui
comme si, brusquement, toutes les étoiles s'éteignaient! Et ce n'est pas important ça!

Il ne put rien dire de plus. Il éclata brusquement
en sanglots. La nuit était tombée. J'avais lâché mes
outils. Je me moquais bien de mon marteau, de mon
boulon, de la soif et de la mort. Il y avait, sur une
étoile, une planète, la mienne, la Terre, un petit
prince consoler! Je le pris dans les bras. Je le berçai. Je lui disais:

-- La fleur que tu aimes n'est pas en danger\ldots{} Je lui dessinerai une muselière, à ton mouton\ldots{} Je te dessinerai une armure pour ta fleur\ldots{} Je\ldots{}

Je ne savais pas trop quoi dire. Je me sentais
très maladroit. Je ne savais comment l'atteindre, où
le rejoindre\ldots{} C'est tellement mystérieux, le pays
des larmes!\\

\textbf{VIII}

J'appris bien vite à mieux connaître cette fleur. Il y
avait toujours eu, sur la planète du petit prince,
des fleurs très simples, ornées d'un seul rang de
pétales, et qui ne tenaient point de place, et qui ne dérangeaient personne. Elles apparaissaient un
matin dans l'herbe, et puis elles s'éteignaient le soir.
Mais celle-là avait germé un jour, d'une graine
apportée d'on ne sait où, et le petit prince avait surveillé de très près cette brindille qui ne ressemblait
pas aux autres brindilles. Ça pouvait être un nouveau genre de baobab. Mais l'arbuste cessa vite de
croître, et commença de préparer une fleur. Le petit
prince, qui assistait à l'installation d'un bouton
énorme, sentait bien qu'il en sortirait une apparition miraculeuse, mais la fleur n'en finissait pas de
se préparer à être belle, à l'abri de sa chambre verte.
Elle choisissait avec soin ses couleurs. Elle s'habillait
lentement, elle ajustait un à un ses pétales. Elle ne
voulait pas sortir toute fripée comme les coquelicots.
Elle ne voulait apparaître que dans le plein rayonnement de sa beauté. Eh! oui. Elle était très coquette!
Sa toilette mystérieuse avait donc duré des jours et
des jours. Et puis voici qu'un matin, justement à
l'heure du lever du soleil, ellc s'était montrée.

Et elle, qui avait travaillé avec tant de précision,
dit en bâillant:

-- Ah! Je me réveille à peine\ldots{} Je vous demande
pardon\ldots{} Je suis encore toute décoiffée\ldots{}

Le petit prince, alors, ne put contenir son admiration:

-- Que vous êtes belle!

-- N'est-ce pas, répondit doucement la fleur. Et je suis née en même temps que le soleil\ldots{}

Le petit prince devina bien qu'elle n'était pas trop modeste, mais elle
était si émouvante!

-- C'est l'heure, je crois, du petit déjeuner, avait-elle bientôt ajouté, auriez-vous la bonté de penser à moi\ldots{}

Et le petit prince, tout confus, ayant été chercher
un arrosoir d'eau fraîche, avait servi la fleur.

\medskip

Ainsi l'avait-elle bien vite tourmenté par sa vanité un peu ombrageuse.
Un jour, par exemple, parlant de ses quatre épines,
elle avait dit au petit prince:

-- Ils peuvent venir, les tigres, avec leurs griffes!

-- Il n'y a pas de tigres sur ma planète, avait
objecté le petit prince, et puis les tigres ne mangent
pas d'herbe.

-- Je ne suis pas une herbe, avait doucement répondu la fleur.

-- Pardonnez -moi\ldots{}

-- Je ne crains rien des tigres, mais j'ai horreur
des courants d'air. Vous n'auriez pas un paravent?

-- Horreur des courants d'air\ldots{} Ce n'est pas de
chance, pour une plante, avait remarqué le petit
prince. Cette fleur est bien compliquée\ldots{}

-- Le soir vous me mettrez sous globe. Il fait très froid chez vous.
C'est mal installé. Là d'où je viens\ldots{}

Mais elle s'était interrompue. Elle était venue sous
forme de graine. Elle n'avait rien pu connaître des autres mondes.
Humiliée de s'être laissé surprendre à préparer un mensonge aussi naïf, elle avait toussé
deux ou trois fois, pour mettre le petit prince dans son tort:

-- Ce paravent?\ldots{}

-- J'allais le chercher mais vous me parliez!

Alors elle avait forcé sa toux pour lui infliger
quand même des remords.

\medskip

Ainsi le petit prince, malgré la bonne volonté de
son amour, avait vite douté d'elle. Il avait pris au
sérieux des mots sans importance, et était devenu
très malheureux.

-- J'aurais dû ne pas l'écouter, me confia-t-il un
jour, il ne faut jamais écouter les fleurs. Il faut les
regarder et les respirer. La mienne embaumait ma
planète, mais je ne savais pas m'en réjouir. Cette histoire de griffes, qui m'avait tellement agacé, eût dû
m'attendrir\ldots{}

Il me confia encore:

-- Je n'ai alors rien su comprendre! J'aurais dû
la juger sur les actes et non sur
les mots. Elle m'embaumait et
m'éclairait. Je n'aurais jamais dû
m'enfuir! J'aurais dû deviner sa
tendresse derrière ses pauvres ruses. Les fleurs sont si contradictoires! Mais j'étais trop jeune
pour savoir l'aimer.\\

\textbf{IX}

Je crois qu'il profita, pour son évasion, d'une
migration d'oiseaux sauvages. Au matin du départ il
mit sa planète bien en ordre. Il ramona soigneusement ses volcans en activité. Il possédait deux volcans en activité.
Et c'était bien commode pour faire
chauffer le petit déjeuner du matin. Il possédait
aussi un volcan éteint. Mais, comme il disait: ``On
ne sait jamais!''. Il ramona donc également le volcan
éteint. S'ils sont bien ramonés, les volcans brûlent
doucement et régulièrement, sans éruptions. Les
éruptions volcaniques sont comme des feux de cheminée. Évidemment sur notre terre nous sommes
beaucoup trop petits pour ramoner nos volcans.
C'est pourquoi ils nous causent des tas d'ennuis.

Le petit prince arracha aussi, avec un peu de
mélancolie, les dernières pousses de baobabs. Il
croyait ne jamais devoir revenir. Mais tous ces travaux familiers lui parurent, ce matin-là, extrêmement doux. Et, quand il arrosa une dernière fois la fleur, et se prépara à la mettre à l'abri sous son
globe, il se découvrit l'envie de pleurer.

-- Adieu, dit-il à la fleur.

Mais elle ne lui répondit pas.

-- Adieu, répéta-t-il.

La fleur toussa. Mais ce n'était pas à cause de son rhume.

-- J'ai été sotte, lui dit-elle enfin. Je te demande
pardon. Tâche d'être heureux.

Il fut surpris par l'absence de reproches. Il restait là tout déconcerté, le globe en l'air. Il ne comprenait pas cette douceur calme.

-- Mais oui, je t'aime, lui dit la fleur. Tu n'en as
rien su, par ma faute. Cela n'a aucune importance.
Mais tu as été aussi sot que moi. Tâche d'être heureux\ldots{} Laisse ce globe tranquille. Je n'en veux plus.

-- Mais le vent\ldots{}

-- Je ne suis pas si enrhumée que ça\ldots{} L'air frais
de la nuit me fera du bien. Je suis une fleur.

-- Mais les bêtes\ldots{}

-- Il faut bien que je supporte deux ou trois chenilles si je veux connaître les papillons. Il paraît que
c'est tellement beau. Sinon qui me rendra visite? Tu
seras loin, toi. Quant aux grosses bêtes, je ne crains
rien. J'ai mes griffes.

Et elle montrait naïvement ses quatre épines. Puis
elle ajouta:

-- Ne traîne pas comme ça, c'est agaçant. Tu as
décidé de partir. Va-t'en.

Car elle ne voulait pas qu'il la vît pleurer. C'était
une fleur tellement orgueilleuse\ldots{}\\

\textbf{X}

Il se trouvait dans la région des astéroïdes 325,
326, 327, 328, 329 et 330. Il commença donc par les
visiter pour y chercher une occupation et pour s'instruire.

Le premier était habité par un roi. Le roi siégeait,
habillé de pourpre et d'hermine, sur un trône très
simple et cependant majestueux.

-- Ah! Voilà un sujet!, s'écria le roi quand il aperçut le petit prince. Et le petit prince se demanda:

-- Comment peut-il me reconnaître puisqu'il ne
m'a encore jamais vu!

Il ne savait pas que, pour les rois, le monde est
très simplifié. Tous les hommes sont des sujets.

-- Approche-toi que je te voie mieux, lui dit le
roi qui était tout fier d'être enfin roi pour quelqu'un.

Le petit prince chercha des yeux où s'asseoir, mais
la planète était tout encombrée par le magnifique
manteau d'hermine. Il resta donc debout, et,
comme il était fatigué, il bâilla.

-- Il est contraire à l'étiquette de bâiller en présence d'un roi, lui dit le monarque. Je te l'interdis.

-- Je ne peux pas m'en empêcher, répondit le
petit prince tout confus. J'ai fait un long voyage et je
n'ai pas dormi\ldots{}

-- Alors, lui dit le roi, je t'ordonne de bâiller. Je n'ai vu personne bâiller depuis des années. Les
bâillements sont pour moi des curiosités. Allons! Bâille encore. C'est un ordre.

-- Ça m'intimide\ldots{} Je ne peux plus\ldots{}, fit le petit
prince tout rougissant.

-- Hum! Hum! répondit le roi. Alors je\ldots{} Je t'ordonne tantôt de bâiller et tantôt de\ldots{}

Il bredouillait un peu et paraissait vexé.

Car le roi tenait essentiellement à ce que son
autorité fût respectée. Il ne tolérait pas la désobéissance. C'était un monarque absolu. Mais, comme il
était très bon, il donnait des ordres raisonnables.

-- Si j'ordonnais, disait-il couramment, si j'ordonnais à un général de se changer en oiseau de mer, et si le général n'obéissait pas, ce ne serait pas la faute du général. Ce serait ma faute.

-- Puis-je m'asseoir? s'enquit timidement le petit prince.

-- Je t'ordonne de t'asseoir, lui répondit le roi, qui ramena majestueusement un pan de son manteau d'hermine.

Mais le petit prince s'étonnait. La planète était
minuscule. Sur quoi le roi pouvait-il bien régner?

-- Sire, lui dit-il\ldots{} je vous demande pardon de
vous interroger\ldots{}

-- Je t'ordonne de m'interroger, se hâta de dire le roi.

-- Sire\ldots{} sur quoi régnez-vous?

-- Sur tout, répondit le roi, avec une grande simplicité.

-- Sur tout?

Le roi d'un geste discret désigna sa planète, les
autres planètes et les étoiles.

-- Sur tout ça? dit le petit prince.

-- Sur tout ça\ldots{}, répondit le roi.

Car non seulement c'était un monarque absolu mais c'était un monarque universel.

-- Et les étoiles vous obéissent?

-- Bien sûr, lui dit le roi. Elles obéissent aussitôt. Je ne tolère pas l'indiscipline.

Un tel pouvoir émerveilla le petit prince. S'il l'avait détenu lui-même, il aurait pu assister, non pas à
quarante-quatre, mais à soixante-douze, ou même à
cent, ou même à deux cents couchers de soleil
dans la même journée, sans avoir jamais à tirer sa
chaise! Et comme il se sentait un peu triste à cause
du souvenir de sa petite planète abandonnée, il
s'enhardit à solliciter une grâce du roi:

-- Je voudrais voir un coucher de soleil\ldots{} Faites-moi plaisir\ldots{} Ordonnez au soleil de se coucher\ldots{}

-- Si j'ordonnais à un général de voler d'une fleur à l'autre à la façon d'un papillon, ou d'écrire une tragédie, ou de se changer en oiseau de mer, et
si le général n'exécutait pas l'ordre reçu, qui, de lui
ou de moi, serait dans son tort?

-- Ce serait vous, dit fermement le petit prince.

-- Exact. Il faut exiger de chacun ce que chacun
peut donner, reprit le roi. L'autorité repose d'abord
sur la raison. Si tu ordonnes à ton peuple d'aller se
jeter à la mer, il fera la révolution. J'ai le droit d'exiger
l'obéissance parce que mes ordres sont raisonnables.

-- Alors mon coucher de soleil? rappela le petit
prince qui jamais n'oubliait une question une fois
qu'il l'avait posée.

-- Ton coucher de soleil, tu l'auras. Je l'exigerai.
Mais j'attendrai, dans ma science du gouvernement,
que les conditions soient favorables.

-- Quand ça sera-t-il? s'informa le petit prince.

-- Hem! hem! lui répondit le roi, qui consulta
d'abord un gros calendrier, hem! hem! ce sera,
vers\ldots{} vers\ldots{} ce sera ce soir vers sept heures quarante! Et tu verras comme je suis bien obéi.

Le petit prince bâilla. Il regrettait son coucher de
soleil manqué . Et puis il s'ennuyait déjà un peu:

-- Je n'ai plus rien à faire ici, dit-il au roi. Je vais
repartir!

-- Ne pars pas, répondit le roi qui était si fier
d'avoir un sujet. Ne pars pas, je te fais ministre!

-- Ministre de quoi?

-- De\ldots{} de la Justice!

-- Mais il n'y a personne à juger!

-- On ne sait pas, lui dit le roi. Je n'ai pas fait
encore le tour de mon royaume. Je suis très vieux, je
n'ai pas de place pour un carrosse, et ça me fatigue
de marcher.

-- Oh! Mais j'ai déjà vu, dit le petit prince qui se
pencha pour jeter encore un coup d'œil sur l'autre
côté de la planète. Il n'y a personne là-bas non
plus\ldots{}

-- Tu te jugeras donc toi-même, lui répondit le
roi. C'est le plus difficile. Il est bien plus difficile de
se juger soi-même que de juger autrui. Si tu réussis à
bien te juger, c'est que tu es un véritable sage.

-- Moi, dit le petit prince, je puis me juger moi-même n'importe où. Je n'ai pas besoin d'habiter ici.

-- Hem! hem! dit le roi, je crois bien que sur ma
planète il y a quelque part un vieux rat. Tu le condamneras à mort de temps en temps. Ainsi, sa vie dépendra de ta justice. Mais tu le gracieras chaque fois
pour l'économiser. Il n'y en a qu'un.

-- Moi, répondit le petit prince, je n'aime pas
condamner à mort, et je crois bien que je m'en vais.

-- Non, dit le roi.

Mais le petit prince, ayant achevé ses préparatifs,
ne voulut point peiner le vieux monarque:

-- Si votre Majesté désirait être obéie ponctuellement, Elle pourrait me donner un ordre raisonnable. Elle pourrait m'ordonner, par exemple, de partir avant une minute. Il me semble que les conditions sont favorables\ldots{}

Le roi n'ayant rien répondu, le petit prince hésita
d'abord, puis, avec un soupir, prit le départ\ldots{}

-- Je te fais mon ambassadeur, se hâta alors de
crier le roi.

Il avait un grand air d'autorité.

-- Les grandes personnes sont bien étranges, se
dit le petit prince, en lui-même, durant son voyage.

\textbf{XI}

La seconde planète était habitée par un vaniteux:

-- Ah! Ah! Voilà la visite d'un admirateur! s'écria de loin le vaniteux dès qu'il aperçut le petit prince.

Car, pour les vaniteux, les autres hommes sont des admirateurs.

-- Bonjour, dit le petit prince. Vous avez un drôle de chapeau.

-- C'est pour saluer, lui répondit le vaniteux. C'est
pour saluer quand on m'acclame. Malheureusement il ne
passe jamais personne par ici.

-- Ah oui? dit le petit prince qui ne comprit pas.

-- Frappe tes mains l'une contre l'autre, conseilla donc le vaniteux.

Le petit prince frappa ses mains l'une contre l'autre. Le vaniteux salua
modestement en soulevant son chapeau.

``Ça, c'est plus amusant que la visite au roi'', se
dit en lui-même le petit prince. Et il recommença de frapper ses mains l'une contre l'autre.
Le vaniteux recommença de saluer en soulevant son chapeau.

Après cinq minutes d'exercice le petit prince
se fatigua de la monotonie du jeu:

-- Et, pour que le chapeau tombe, demanda-t-il,
que faut-il faire?

Mais le vaniteux ne l'entendit pas. Les vaniteux
n'entendent jamais que les louanges.

-- Est-ce que tu m'admires vraiment beaucoup?
demanda-t-il au petit prince.

-- Qu'est-ce que signifie ``admirer''?

-- ``Admirer'' signifie reconnaître que je suis
l'homme le plus beau, le mieux habillé, le plus riche
et le plus intelligent de la planète.

-- Mais tu es seul sur ta planète!

-- Fais-moi ce plaisir. Admire-moi quand même!

-- Je t'admire, dit le petit prince, en haussant un
peu les épaules, mais en quoi cela peut-il bien t'intéresser?

Et le petit prince s'en fut.

``Les grandes personnes sont décidément bien
bizarres'', se dit-il simplement en lui-même durant
son voyage.\\

\textbf{XII}

La planète suivante était habitée par un buveur.
Cette visite fut très courte mais elle plongea le petit
prince dans une grande mélancolie:

-- Que fais-tu là? dit-il au buveur, qu'il trouva installé en silence devant une collection de bouteilles
vides et une collection de bouteilles pleines.

-- Je bois, répondit le buveur, d'un air lugubre.

-- Pourquoi bois-tu? lui demanda le petit prince.

-- Pour oublier, répondit le buveur.

-- Pour oublier quoi? s'enquit le petit prince qui
déjà le plaignait.

-- Pour oublier que j'ai honte, avoua le buveur en baissant la tête.

-- Honte de quoi? s'informa le petit prince qui
désirait le secourir.

-- Honte de boire!, acheva le buveur qui s'enferma définitivement dans le silence.

Et le petit prince s'en fut, perplexe.

``Les grandes personnes sont décidément très très
bizarres'', se disait-il en lui-même durant le voyage.\\

\textbf{XIII}

La quatrième planète était celle du businessman.
Cet homme était si occupé qu'il ne leva même pas la
tête à l'arrivée du petit prince.

-- Bonjour, lui dit celui-ci. Votre cigarette est
éteinte.

-- Trois et deux font cinq. Cinq et sept douze.
Douze et trois quinze. Bonjour. Quinze et sept vingt-deux. Vingt-deux et six vingt-huit. Pas le temps de la
rallumer. Vingt-six et cinq trente et un. Ouf! Ça fait
donc cinq cent un millions six cent vingt-deux mille
sept cent trente et un.

-- Cinq cents millions de quoi?

-- Hein? Tu es toujours là? Cinq cent un millions de\ldots{} je ne sais plus\ldots{} j'ai tellement de travail!
Je suis sérieux, moi, je ne m'amuse pas à des balivernes! Deux et cinq sept\ldots{}

-- Cinq cent un millions de quoi?, répéta le
petit prince qui jamais de sa vie n'avait renoncé à
une question, une fois qu'il l'avait posée.

Le businessman leva la tête:

-- Depuis cinquante-quatre ans que j'habite cette
planète-ci, je n'ai été dérangé que trois fois. La première fois ç'a été, il y a vingt-deux ans, par un hanneton qui était tombé dieu sait d'où. Il répandait un
bruit épouvantable, et j'ai fait quatre erreurs dans
une addition. La seconde fois ç'a été, il y a onze ans,
par une crise de rhumatisme. Je manque d’exercice.
Je n'ai pas le temps de flâner. Je suis sérieux, moi. La
troisième fois\ldots{} la voici! Je disais donc cinq cent un
millions\ldots{}

-- Millions de quoi?

Le businessman comprit qu'il n'était point d'espoir de paix:

-- Millions de ces petites choses que l'on voit quelquefois dans le ciel.

-- Des mouches?

-- Mais non, des petites choses qui brillent.

-- Des abeilles?

-- Mais non. Des petites choses dorées qui font rêvasser les fainéants. Mais je suis sérieux, moi! Je
n'ai pas le temps de rêvasser.

-- Ah! des étoiles?

-- C'est bien ça. Des étoiles.

-- Et que fais-tu de cinq cents millions d'étoiles?

-- Cinq cent un millions six cent vingt-deux mille
sept cent trente et un. Je suis sérieux, moi, je suis
précis.

-- Et que fais-tu de ces étoiles?

-- Ce que j'en fais?

-- Oui.

-- Rien. Je les possède.

-- Tu possèdes les étoiles?

-- Oui.

-- Mais j'ai déjà vu un roi qui\ldots{}

-- Les rois ne possèdent pas. Ils ``règnent'' sur. C'est très différent.

-- Et à quoi cela te sert-il de posséder les étoiles?

-- Ça me sert à être riche.

-- Et à quoi cela te sert-il d'être riche?

-- À acheter d'autres étoiles, si quelqu'un en trouve.

``Celui-là, se dit en lui-même le petit prince, il raisonne un peu comme mon ivrogne.''

Cependant il posa encore des questions:

-- Comment peut-on posséder les étoiles?

-- À qui sont-elles? riposta, grincheux, le businessman.

-- Je ne sais pas. À personne.

-- Alors elles sont à moi, car j'y ai pensé le premier.

-- Ça suffit?

-- Bien sûr. Quand tu trouves un diamant qui
n'est à personne, il est à toi. Quand tu trouves une
île qui n'est à personne, elle est à toi. Quand tu as
une idée le premier, tu la fais breveter: elle est à toi.
Et moi je possède les étoiles, puisque jamais personne avant moi n'a songé à les posséder.

-- Ça c'est vrai, dit le petit prince. Et qu'en fais-tu?

-- Je les gère. Je les compte et je les recompte, dit
le businessman. C'est difficile. Mais je suis un homme sérieux!

Le petit prince n'était pas satisfait encore.

-- Moi, si je possède un foulard, je puis le mettre
autour de mon cou et l'emporter. Moi, si je possède
une fleur, je puis cueillir ma fleur et l'emporter.
Mais tu ne peux pas cueillir les étoiles!

-- Non, mais je puis les placer en banque.

-- Qu'est-ce que ça veut dire?

-- Ça veut dire que j'écris sur un petit papier le
nombre de mes étoiles. Et puis j'enferme à clef ce
papier-là dans un tiroir.

-- Et c'est tout?

-- Ça suffit!

``C'est amusant'', pensa le petit prince. ``C'est assez
poétique. Mais ce n'est pas très sérieux''.

Le petit prince avait sur les choses sérieuses des idées très différentes des idées des grandes personnes.

-- Moi, dit-il encore, je possède une fleur que j'arrose tous les jours. Je possède trois volcans que je
ramone toutes les semaines. Car je ramone aussi
celui qui est éteint. On ne sait jamais. C'est utile à
mes volcans, et c'est utile à ma fleur, que je les possède. Mais tu n'es pas utile aux étoiles\ldots{}

Le businessman ouvrit la bouche mais ne trouva
rien à répondre, et le petit prince s'en fut.

``Les grandes personnes sont décidément tout à
fait extraordinaires'', se disait-il simplement en lui-même durant le voyage.\\

\textbf{XIV}

La cinquième planète était très curieuse. C'était la
plus petite de toutes. Il y avait là juste assez de place
pour loger un réverbère et un allumeur de réverbères. Le petit prince ne parvenait pas à s'expliquer à
quoi pouvaient servir, quelque part dans le ciel,
sur une planète sans maison ni population, un
réverbère et un allumeur de réverbères. Cependant
il se dit en lui-même:

``Peut-être bien que cet homme est absurde.
Cependant il est moins absurde que le roi, que le
vaniteux, que le businessman et que le buveur. Au
moins son travail a-t-il un sens. Quand il allume son
réverbère, c'est comme s'il faisait naître une étoile
de plus, ou une fleur. Quand il éteint son réverbère,
ça endort la fleur ou l'étoile. C'est une occupation très jolie. C'est véritablement utile puisque c'est
joli.

Lorsqu'il aborda la planète, il salua respectueusement l'allumeur:

-- Bonjour. Pourquoi viens-tu d'éteindre ton réverbère?

-- C'est la consigne, répondit l'allumeur. Bonjour

-- Qu'est-ce que la consigne?

-- C'est d'éteindre mon réverbère. Bonsoir.

Et il le ralluma.

-- Mais pourquoi viens-tu de le rallumer?

-- C'est la consigne, répondit l'allumeur.

-- Je ne comprends pas, dit le petit prince.

-- Il n'y a rien à comprendre, dit l'allumeur. La
consigne c'est la consigne. Bonjour.

Et il éteignit son réverbère.

Puis il s'épongea le front avec un mouchoir à carreaux rouges.

-- Je fais là un métier terrible. C'était raisonnable
autrefois. J'éteignais le matin et j'allumais le soir.
J'avais le reste du jour pour me reposer, et le reste
de la nuit pour dormir\ldots{}

-- Et, depuis cette époque, la consigne a changé?

-- La consigne n'a pas changé, dit l'allumeur.
C'est bien là le drame! La planète d'année en année a tourné de plus en plus vite, et la consigne n'a
pas changé!

-- Alors? dit le petit prince.

-- Alors maintenant qu'elle fait un tour par
minute, je n'ai plus une seconde de repos. J'allume
et j'éteins une fois par minute!

-- Ça c'est drôle! Les jours chez toi durent une
minute!

-- Ce n'est pas drôle du tout, dit l'allumeur. Ça
fait déjà un mois que nous parlons ensemble.

-- Un mois?

-- Oui. Trente minutes. Trente jours! Bonsoir.

Et il ralluma son réverbère.

Le petit prince le regarda et il aima cet allumeur
qui était tellement fidèle à la consigne. Il se souvint
des couchers de soleil que lui-même allait autrefois
chercher, en tirant sa chaise. Il voulut aider son
ami:

-- Tu sais\ldots{} je connais un moyen de te reposer
quand tu voudras\ldots{}

-- Je veux toujours, dit l'allumeur.

Car on peut être, à la fois, fidèle et paresseux.

Le petit prince poursuivit:

-- Ta planète est tellement petite que tu en fais le
tour en trois enjambées. Tu n'as qu'à marcher assez
lentement pour rester toujours au soleil. Quand tu
voudras te reposer tu marcheras\ldots{} et le jour durera
aussi longtemps que tu voudras.

-- Ça ne m'avance pas à grand-chose, dit l'allumeur. Ce que j'aime dans la vie, c'est dormir.

-- Ce n'est pas de chance, dit le petit prince.

-- Ce n'est pas de chance, dit l'allumeur. Bonjour.

Et il éteignit son réverbère.

``Celui-là'', se dit le petit prince, tandis qu'il poursuivait plus loin son voyage, ``celui-là serait méprisé
par tous les autres, par le roi, par le vaniteux, par le
buveur, par le businessman. Cependant c'est le seul
qui ne me paraisse pas ridicule. C'est, peut-être, parce qu'il s'occupe d'autre chose que de soi-même''.

Il eut un soupir de regret et se dit encore:

``Celui-là est le seul dont j'eusse pu faire mon ami.
Mais sa planète est vraiment trop petite. Il n'y a pas
de place pour deux\ldots{}''

Ce que le petit prince n'osait pas s'avouer, c'est
qu'il regrettait cette planète bénie à cause, surtout,
des mille quatre cent quarante couchers de soleil
par vingt-quatre heures!\\

\textbf{XV}

La sixième planète était une planète dix fois plus
vaste. Elle était habitée par un vieux monsieur qui
écrivait d'énormes livres.

-- Tiens! voilà un explorateur!, s'écria-t-il, quand
il aperçut le petit prince.

Le petit prince s'assit sur la table et souffla un
peu. Il avait déjà tant voyagé!

-- D'où viens-tu? lui dit le vieux monsieur.

-- Quel est ce gros livre? dit le petit prince. Que
faites-vous ici?

-- Je suis géographe, dit le vieux monsieur.

-- Qu'est-ce qu'un géographe?

-- C'est un savant qui connaît où se trouvent les
mers, les fleuves, les villes, les montagnes et les
déserts.

-- Ça c'est bien intéressant, dit le petit prince. Ça
c'est enfin un véritable métier!

Et il jeta un coup d’œil autour de lui sur la planète du géographe. Il n'avait jamais vu encore une planète aussi majestueuse.

-- Elle est bien belle, votre planète. Est-ce qu'il y a
des océans?

-- Je ne puis pas le savoir, dit le géographe.

-- Ah! (Le petit prince était déçu). Et des montagnes?

-- Je ne puis pas le savoir, dit le géographe.

-- Et des villes et des fleuves et des déserts?

-- Je ne puis pas le savoir non plus, dit le géographe.

-- Mais vous êtes géographe!

-- C'est exact, dit le géographe, mais je ne suis
pas explorateur. Je manque absolument d'explorateurs. Ce n'est pas le géographe qui va faire le
compte des villes, des fleuves, des montagnes, des mers, des océans et des déserts. Le géographe est
trop important pour flâner. Il ne quitte pas son
bureau. Mais il y reçoit les explorateurs. Il les interroge, et il prend en note leurs souvenirs. Et si les
souvenirs de l'un d'entre eux lui paraissent intéressants, le géographe fait faire une enquête sur la
moralité de l'explorateur.

-- Pourquoi ça?

-- Parce qu'un explorateur qui mentirait entraînerait des catastrophes dans les livres de géographie. Et aussi un explorateur qui boirait trop.

-- Pourquoi ça? fit le petit prince.

-- Parce que les ivrognes voient double. Alors le
géographe noterait deux montagnes, là où il n'y en a
qu'une seule.

-- Je connais quelqu'un, dit le petit prince, qui
serait mauvais explorateur.

-- C'est possible. Donc, quand la moralité de
l'explorateur paraît bonne, on fait une enquête sur
sa découverte.

-- On va voir?

-- Non. C'est trop compliqué. Mais on exige de
l'explorateur qu'il fournisse des preuves. S'il s'agit
par exemple de la découverte d'une grosse montagne, on exige qu'il en rapporte de grosses
pierres.

Le géographe soudain s'émut.

-- Mais toi, tu viens de loin! Tu es explorateur! Tu
vas me décrire ta planète!

Et le géographe, ayant ouvert son registre, tailla
son crayon. On note d'abord au crayon les récits des
explorateurs. On attend, pour noter à l'encre, que
l'explorateur ait fourni des preuves.

-- Alors? interrogea le géographe.

-- Oh! chez moi, dit le petit prince, ce n'est pas
très intéressant, c'est tout petit. J'ai trois volcans.
Deux volcans en activité, et un volcan éteint. Mais
on ne sait jamais.

-- On ne sait jamais, dit le géographe.

-- J'ai aussi une fleur.

-- Nous ne notons pas les fleurs, dit le géographe.

-- Pourquoi ça! c'est le plus joli!

-- Parce que les fleurs sont éphémères.

-- Qu'est-ce que signifie: ``éphémère''?

-- Les géographies, dit le géographe, sont les
livres les plus sérieux de tous les livres. Elles ne se
démodent jamais. Il est très rare qu'une montagne
change de place. Il est très rare qu'un océan se vide
de son eau. Nous écrivons des choses éternelles.

-- Mais les volcans éteints peuvent se réveiller,
interrompit le petit prince. Qu'est-ce que signifie:
``éphémère''?

-- Que les volcans soient éteints ou soient
éveillés, ça revient au même pour nous autres, dit le
géographe. Ce qui compte pour nous, c'est la montagne. Elle ne change pas.

-- Mais qu'est-ce que signifie ``éphémère''? répéta le petit prince qui, de sa vie, n'avait renoncé à
une question, une fois qu'il l'avait posée.

-- Ça signifie ``qui est menacé de disparition prochaine''.

-- Ma fleur est menacée de disparition prochaine?

-- Bien sûr.

``Ma fleur est éphémère'', se dit le petit prince, ``et elle n'a que quatre épines pour se défendre contre
le monde! Et je l'ai laissée toute seule chez moi!''

Ce fut là son premier mouvement de regret. Mais
il reprit courage:

-- Que me conseillez-vous d'aller visiter? demanda-t-il.

-- La planète Terre, lui répondit le géographe.
Elle a une bonne réputation\ldots{}

Et le petit prince s'en fut, songeant à sa fleur.\\

\textbf{XVI}

La septième planète fut donc la Terre.

La Terre n'est pas une planète quelconque! On y
compte cent onze rois (en n'oubliant pas, bien sûr,
les rois nègres), sept mille géographes, neuf cent
mille businessmen, sept millions et demi d'ivrognes,
trois cent onze millions de vaniteux, c'est-à-dire
environ deux milliards de grandes personnes.

Pour vous donner une idée des dimensions de la
Terre je vous dirai qu'avant l'invention de l'électricité
on y devait entretenir, sur l'ensemble des six continents, une véritable armée de quatre cent soixante
deux mille cinq cent onze allumeurs de réverbères.

Vu d'un peu loin ça faisait un effet splendide. Les
mouvements de cette armée étaient réglés comme
ceux d'un ballet d'opéra. D'abord venait le tour des
allumeurs de réverbères de Nouvelle-Zélande et
d'Australie. Puis ceux-ci, ayant allumé leurs lampions,
s'en allaient dormir. Alors entraient à leur tour dans la
danse les allumeurs de réverbères de Chine et de Sibérie.
Puis eux aussi s'escamotaient dans les coulisses.
Alors venait le tour des allumeurs de réverbères de
Russie et des Indes. Puis de ceux d'Afrique et d'Europe. Puis de ceux d'Amérique du Sud. Puis de ceux
d'Amérique du Nord. Et jamais ils ne se trompaient
dans leur ordre d'entrée en scène. C'était grandiose.

Seuls, l'allumeur de l'unique réverbère du pôle
Nord, et son confrère de l'unique réverbère du pôle
Sud, menaient des vies d'oisiveté et de nonchalance: ils travaillaient deux fois par an.\\

\textbf{XVII}

Quand on veut faire de l'esprit, il arrive que l'on
mente un peu. Je n'ai pas été très honnête en vous
parlant des allumeurs de réverbères. Je risque de
donner une fausse idée de notre planète à ceux qui
ne la connaissent pas. Les hommes occupent très
peu de place sur la Terre. Si les deux milliards d'habitants qui peuplent la Terre se tenaient debout et
un peu serrés, comme pour un meeting, ils logeraient aisément sur une place publique de vingt
milles de long sur vingt milles de large. On pourrait
entasser l'humanité sur le moindre petit îlot du
Pacifique.

Les grandes personnes, bien sûr, ne vous croiront
pas. Elles s'imaginent tenir beaucoup de place. Elles
se voient importantes comme des baobabs. Vous
leur conseillerez donc de faire le calcul. Elles adorent les chiffres: ça leur plaira. Mais ne perdez pas
votre temps à ce pensum. C'est inutile. Vous avez
confiance en moi.

Le petit prince, une fois sur Terre, fut donc bien
surpris de ne voir personne. Il avait déjà peur de
s'être trompé de planète, quand un anneau couleur
de lune remua dans le sable.

-- Bonne nuit, fit le petit prince à tout hasard.

-- Bonne nuit, fit le serpent.

-- Sur quelle planète suis"-je tombé? demanda le petit prince.

-- Sur la Terre, en Afrique, répondit le serpent.

-- Ah!\ldots{} Il n'y a donc personne sur la Terre?

-- Ici c'est le désert. Il n'y a personne dans les
déserts. La Terre est grande, dit le serpent.

Le petit prince s'assit sur une pierre et leva les yeux vers le ciel:

-- Je me demande, dit"-il, si les étoiles sont éclairées
afin que chacun puisse un jour retrouver la sienne. Regarde ma planète. Elle est juste au"-dessus de
nous\ldots{} Mais comme elle est loin!

-- Elle est belle, dit le serpent. Que viens"-tu faire ici?

-- J'ai des difficultés avec une fleur, dit le petit prince.

-- Ah!, fit le serpent.

Et ils se turent.

-- Où sont les hommes? reprit enfin le petit prince. On est un peu seul dans le désert\ldots{}

-- On est seul aussi chez les hommes, dit le serpent.

Le petit prince le regarda longtemps:

-- Tu es une drôle de bête, lui dit"-il enfin, mince
comme un doigt\ldots{}

-- Mais je suis plus puissant que le doigt d'un roi, dit le serpent.

Le petit prince eut un sourire:

-- Tu n'es pas bien puissant\ldots{} tu n'as même pas de
pattes\ldots{} tu ne peux même pas voyager\ldots{}

-- Je puis t'emporter plus loin qu'un navire, dit le serpent.

Il s'enroula autour de la cheville du petit prince, comme un bracelet d'or:

-- Celui que je touche, je le rends à la terre dont il
est sorti, dit"-il encore. Mais tu es pur et tu viens
d'une étoile\ldots{}

Le petit prince ne répondit rien.

-- Tu me fais pitié, toi si faible, sur cette Terre de
granit. Je puis t'aider un jour si tu regrettes trop ta
planète. Je puis\ldots{}

-- Oh! J'ai très bien compris, fit le petit prince,
mais pourquoi parles"-tu toujours par énigmes?

-- Je les résous toutes, dit le serpent.

Et ils se turent.\\

\textbf{XVIII}

Le petit prince traversa le désert et ne rencontra
qu'une fleur. Une fleur à trois pétales, une fleur de
rien du tout\ldots{}

-- Bonjour, dit le petit prince.

-- Bonjour, dit la fleur.

-- Où sont les hommes?, demanda poliment le petit prince.

La fleur, un jour, avait vu passer une caravane:

-- Les hommes? Il en existe, je crois, six ou sept. Je
les ai aperçus il y a des années. Mais on ne sait
jamais où les trouver. Le vent les promène. Ils manquent de racines, ça les gêne beaucoup.

-- Adieu, fit le petit prince.

-- Adieu, dit la fleur.\\

\textbf{XIX}

Le petit prince fit l'ascension d'une haute montagne. Les seules montagnes qu'il eût jamais
connues étaient les trois volcans qui lui arrivaient au
genou. Et il se servait du volcan éteint comme d'un
tabouret.

-- D'une montagne haute comme celle"-ci, se dit"-il donc, j'apercevrai d'un coup toute la planète et tous les hommes\ldots{}

Mais il n'aperçut rien que des aiguilles de roc bien aiguisées.

-- Bonjour, dit"-il à tout hasard.

-- Bonjour\ldots{} Bonjour\ldots{} Bonjour\ldots{}, répondit l'écho.

-- Qui êtes"-vous? dit le petit prince.

-- Qui êtes"-vous\ldots{} qui êtes"-vous\ldots{} qui êtes"-vous\ldots{},
répondit l'écho.

-- Soyez mes amis, je suis seul, dit"-il.

-- Je suis seul\ldots{} je suis seul\ldots{} je suis seul\ldots{}, répondit l'écho.

``Quelle drôle de planète! pensa"-t"-il alors. Elle est
toute sèche, et toute pointue et toute salée. Et les
hommes manquent d'imagination. Ils répètent ce
qu'on leur dit\ldots{} Chez moi j'avais une fleur: elle
parlait toujours la première\ldots{}''\\

\textbf{XX}

Mais il arriva que le petit prince, ayant longtemps
marché à travers les sables, les rocs et les neiges,
découvrit enfin une route. Et les routes vont toutes
chez les hommes.

-- Bonjour, dit"-il.

C'était un jardin fleuri de roses.

-- Bonjour, dirent les roses.

Le petit prince les regarda. Elles ressemblaient
toutes à sa fleur.

-- Qui êtes"-vous? leur demanda"-t"-il, stupéfait.

-- Nous sommes des roses, dirent les roses.

-- Ah!, fit le petit prince\ldots{}

Et il se sentit très malheureux. Sa fleur lui avait
raconté qu'elle était seule de son espèce dans l'univers. Et voici qu'il en était cinq mille, toutes semblables, dans un seul jardin!

-- Elle serait bien vexée, se dit"-il, si elle voyait ça\ldots{} elle tousserait énormément et ferait semblant de
mourir pour échapper au ridicule. Et je serais bien
obligé de faire semblant de la soigner, car, sinon,
pour m'humilier moi aussi, elle se laisserait vraiment mourir\ldots{}''

Puis il se dit encore: ``Je me croyais riche d'une
fleur unique, et je ne possède qu'une rose ordinaire. Ça et mes trois volcans qui m'arrivent au
genou, et dont l'un, peut"-être, est éteint pour toujours, ça ne fait pas de moi un bien grand prince\ldots{}''

Et, couché dans l'herbe, il pleura.\\
}


\ParallelRText{
\textbf{I}

Certa vez, quando eu tinha seis anos, vi uma linda imagem num livro
chamado \emph{Histórias vividas}, sobre a floresta virgem. A imagem
representava uma jiboia deglutindo uma fera. Eis a cópia do desenho.

O livro dizia: ``As jiboias engolem uma presa inteira sem mastigar.
Depois, elas não podem se mexer e dormem durante os seis meses de
digestão''.

Refleti muito sobre as aventuras da floresta e, com um lápis de cor,
consegui traçar meu primeiro desenho. Meu desenho número 1. Ele era
assim:

Mostrei minha obra-prima para as pessoas grandes e perguntei se meu
desenho lhes dava medo.

Elas responderam: ``Mas por que um chapéu daria medo?''.

Meu desenho não representava um chapéu. Ele representava uma jiboia
digerindo um elefante. Então desenhei o interior da jiboia para que as
pessoas grandes pudessem entender. Elas sempre precisam de explicações.
Meu desenho número 2 era assim:

As grandes pessoas me aconselharam a deixar de lado os desenhos das
jiboias abertas ou fechadas e me dedicar mais à geografia, à história,
ao cálculo e à gramática. Foi assim que abandonei minha carreira de
pintor. Eu tinha sido desencorajado por causa do fracasso do meu desenho
número 1 e do meu desenho número 2. As grandes pessoas nunca entendem
nada sozinhas, e às vezes é cansativo para as crianças ter que dar
explicações o tempo inteiro, sem parar\ldots{}

Por causa disso escolhi outra profissão e aprendi a pilotar avião. Voei
um pouco pra todo canto no mundo. É claro que a geografia me serviu
bastante. Só de bater os olhos, eu sabia distinguir a China do Arizona.
Isso é muito útil quando estamos perdidos à noite.

Durante minha vida, fiz vários contatos com um monte de gente séria.
Convivi com as pessoas grandes. As vi de muito perto. O que não melhorou
minha opinião.

Quando encontrava uma que me parecia um pouco lúcida, eu fazia o teste
do meu desenho número 1, que sempre guardei comigo. Queria saber se ela
era verdadeiramente compreensiva. Mas ela sempre me respondia: ``É um
chapéu''. Então eu não lhe falava nem sobre as jiboiais, nem sobre as
florestas virgens, nem sobre as estrelas. Punha-me ao seu alcance.
Falava-lhe sobre bridge, golfe, política e gravatas. E a pessoa grande
ficava contente por conhecer um homem tão inteligente\ldots{}\\

\textbf{II}

Por isso, vivi sozinho, sem ter ninguém com quem conversar de verdade,
até passar por uma pane no deserto do Saara, há seis anos. Algo havia
quebrado no meu motor. E, como eu não tinha a companhia de um mecânico
nem de passageiros, preparei-me para arriscar a empreender sozinho um
difícil conserto. Para mim, era uma questão de vida ou morte. A água que
eu tinha dava pra oito dias, e olhe lá.

Na primeira noite dormi na areia, há mil milhas de qualquer terra
habitada. Eu estava muito mais isolado do que um náufrago em uma jangada
no meio do oceano. Imagina minha surpresa quando, ao nascer do dia, uma
vozinha estranha veio me acordar. Ela dizia:

- Por favor... desenha pra mim um carneiro!

- O quê?

- Desenha pra mim um carneiro...

Levantei-me imediatamente, como se tivesse sido acertado por um raio.
Esfreguei os olhos. Olhei bem. E avistei um rapazinho totalmente
incrível que me observava com um ar grave. Esse foi o melhor retrato
que, mais tarde, consegui fazer dele. É claro que o meu desenho é muito
menos elegante do que o modelo. Não tenho culpa. Com seis anos de idade,
eu tinha sido desencorajado em minha carreira de pintor pelas pessoas
adultas, e nunca aprendi a desenhar nada a não ser jiboias fechadas e
jiboias abertas.

Então olhei essa aparição com os olhos arregalados de surpresa. Não se
esqueçam de que eu estava há mil milhas de qualquer região habitada.
Ora, esse rapazinho não parecia surpreso, nem morto de cansaço, nem
morto de fome, nem morto de sede, nem morto de medo. Não tinha
absolutamente a aparência de uma criança perdida no meio do deserto a
mais de mil milhas de qualquer região habitada. Quando finalmente
consegui falar, perguntei-lhe:

- Mas... o que você faz aqui?

Ele então me repetiu, lentamente, como se se tratasse de algo muito
sério:

- Por favor... desenha pra mim um carneiro...

Quando o mistério é muito impressionante, não ousamos desobedecer. Por
mais absurdo que aquilo me parecesse a mil de milhas de qualquer lugar
habitado e correndo risco de vida, tirei do bolso uma folha de papel e
uma caneta. Mas daí me lembrei de que havia estudado geografia,
história, cálculo e gramática e contei ao rapaz (com um pouco de mal
humor) que eu não sabia desenhar. Ele me respondeu:

- Não tem problema. Desenha pra mim um carneiro.

Como eu nunca tinha desenhado um carneiro antes, refiz pra ele um dos
dois únicos desenhos que eu conseguia fazer. O da jiboia fechada. E
fiquei perplexo ao ouvi-lo responder:

- Não! Não! Eu não quero um elefante dentro de uma jiboia. Uma jiboia é
muito perigosa e um elefante ocupa espaço demais. Tudo é pequeno onde eu
moro. Quero um carneiro. Desenha pra mim um carneiro.

Então desenhei.

Ele olhou atento e disse:

- Não! Esse daí já está muito doente. Faz outro.

Desenhei.

Meu amigo sorriu gentilmente, com indulgência:

- Olha direito... Isso não é um carneiro, é um cabrito. Tem chifres...

Refiz outra vez o desenho:

Mas ele foi recusado, assim como os anteriores:

- Esse daí tá velho demais. Eu quero um carneiro que viva por muito
tempo.

Então, por falta de paciência, como eu tinha pressa em começar a
desmontar meu motor, rabisquei esse desenho:

E lancei:

- Isto é uma caixa. O carneiro que você está pedindo está dentro dela.

Mas fui tomado de surpresa ao perceber que o rosto do meu jovem juiz se
iluminava:

- É exatamente assim que eu queria! Você acha que esse carneiro precisa
de muita grama?

- Por quê?

- Porque onde eu moro é pequeno...

- Isso certamente basta. Estou te dando um carneirinho bem pequeno.

Ele inclinou a cabeça por cima do desenho:

- Não é tão pequeno assim... Olha! Ele dormiu...

E foi assim que conheci o pequeno príncipe.\\

\textbf{III}

Demorei muito pra entender de onde ele vinha. O pequeno príncipe, que me
fazia muitas perguntas, parecia nunca ouvir as minhas. Foram as palavras
pronunciadas ao acaso que aos poucos me revelaram tudo. Quando avistou
pela primeira vez meu avião (não irei desenhá-lo aqui, porque é um
desenho muito complicado pra mim), ele perguntou:

- O que é isso?

- Não é uma coisa qualquer. Isso voa. É um avião. É o meu avião.

Eu estava muito orgulhoso em lhe contar que eu voava. Ele exclamou:

- Como assim? Então você caiu do céu!

- Sim -- respondi modestamente.

- Que engraçado!...

E o pequeno príncipe deu uma bela gargalhada que me deixou furioso.
Gosto que levem a sério minhas desgraças. Depois continuou:

- Você também vem do céu! De que planeta você é?

Logo vislumbrei um brilho no mistério de sua presença e perguntei
bruscamente:

- Então você vem de um outro planeta?

Mas ele não respondeu. Balançava a cabeça pausadamente enquanto
observava meu avião:

- Pensando bem, você não pode estar vindo de muito longe nisso daí...

E se lançou num devaneio que tomou um bom tempo. Depois, tirando meu
carneiro do bolso, mergulhou na contemplação do seu tesouro.

Imagine o quanto fiquei intrigado com essa semiconfidência sobre
``outros planetas''. Por isso me esforcei em descobrir mais sobre o
assunto:

- De onde você vem, meu querido? Onde é seu lugar? Pra onde pretende
levar o meu carneiro?

Ele me respondeu após um silêncio meditativo:

- O melhor de tudo, com a caixa que você me deu, é que ela vai lhe
servir de casa à noite.

- Claro. E se você for um bom menino eu também vou lhe dar uma corda
para amarrá-lo durante o dia. E uma estaca.

A proposta pareceu chocar o pequeno príncipe:

- Amarrá-lo? Mas que ideia é essa?

- É que se você não o amarrar, ele pode ir pra qualquer lugar e se
perder.

Meu amigo começou a gargalhar novamente:

- Mas onde você pensa que ele vai?

- Não importa. Basta sair andando pra frente...

Então o pequeno príncipe observou com seriedade:

- Não tem problema algum, é tão pequeno lá onde eu moro!

E, talvez um pouco melancólico, acrescentou:

- Seguindo sempre em frente não dá pra ir muito longe...\\

\textbf{IV}

Foi assim que aprendi uma segunda coisa muito importante: seu planeta de
origem não era tão maior do que uma casa!

Não que isso fosse surpresa pra mim. Eu sabia que além dos grandes
planetas como a Terra, Júpiter, Marte, Vênus, os quais nomeamos, existem
centenas de outros que, de tão pequenos, temos dificuldade em visualizar
no telescópio. Quando um astrônomo descobre um deles, dá-lhe como nome
um número. Chama-o, por exemplo: ``asteroide 325''.

Tenho bons motivos para acreditar que o planeta de onde o pequeno
príncipe vinha era o asteroide B 612. Esse esteroide fora avistado
apenas uma vez no telescópio, em 1909, por um astrônomo turco.

Ele havia feito uma bela demonstração de sua descoberta num congresso
internacional de astronomia. Mas ninguém lhe botou muita fé por causa
das roupas que usava. As pessoas grandes são assim.

Felizmente, para a reputação do asteroide B 612, um ditador turco impôs
ao seu povo, sob pena de morte, que se vestissem como europeus. O
astrônomo refez sua demonstração em 1920 com uma roupa muito elegante. E
dessa vez todos concordaram com ele.

Se lhes dou detalhes a respeito do asteroide B 621 e se lhes confio seu
número é por causa das pessoas grandes. As grandes pessoas adoram
números. Quando a gente lhes conta sobre um novo amigo, elas nunca
querem saber o mais importante. Nunca perguntam: ``Como é a sua voz?
Quais são seus brinquedos de preferência? Ele coleciona borboletas?''.
Em vez disso, perguntam: ``Quantos anos ele tem? Quantos irmãos ele tem?
Quanto ele pesa? Quanto o seu pai ganha?''. E só depois disso acreditam
conhecê-lo. Se você disser às pessoas grandes: ``Vi uma casa linda com
tijolos cor-de-rosa, com gerânios nas janelas e pombos no telhado....'',
elas nunca vão conseguir imaginar essa casa. É preciso lhes dizer: ``Vi
uma casa de cem mil francos''. Daí exclamam: ``Que linda!''.

Por isso, se você disser a elas ``A prova de que o pequeno príncipe
existiu é que ele estava deslumbrante, ele ria e queria um carneiro.
Quando uma pessoa quer um carneiro é a prova de que ela existe'', eles
darão de ombros e te tratarão como criança! Mas, se você disser: ``O
planeta de onde ele veio é o asteroide B 612'', elas ficarão convencidas
e não vão te encher de perguntas. Elas são assim. Mas ninguém precisa
odiá-las por causa isso. As crianças devem ser muito tolerantes com as
pessoas grandes.

Mas, é claro que, nós, que compreendemos a vida, não levamos os números
a sério! Eu teria adorado começar essa história como um conto de fadas.
Adoraria ter escrito:

``Era uma vez um pequeno príncipe que morava num planeta que era quase
do seu tamanho, e que precisava de um amigo''... Para os que compreendem
a vida, isso teria soado muito mais verdadeiro.

Pois eu não gosto que leiam meu livro superficialmente. Já tenho tanta
dor de cabeça contando essas lembranças. Faz seis anos que meu amigo
partiu com seu carneiro. Se tento descrevê-lo aqui é para não o
esquecer. É muito triste esquecer um amigo. Nem todo mundo teve um
amigo. E eu corro o risco de passar a ser como as pessoas grandes, que
não se interessam por outra coisa além dos números. É por isso que
comprei uma caixa de giz de cera e lápis. É duro voltar a desenhar com a
minha idade, quando o máximo que desenhamos na vida foi uma jiboia
aberta, aos 6 anos de idade! É claro que tentarei fazer os retratos mais
fiéis possíveis. Mas não sei se vou conseguir. Às vezes um desenho até
passa mas o outro não dá certo. Engano-me também quanto ao tamanho. Aqui
o pequeno príncipe está muito grande. Já ali, pequeno demais. Hesito
sobre a cor de sua roupa. Daí me esforço um pouco ali, um pouco aqui, do
jeito que dá. Enganei-me quanto aos detalhes mais importantes. Mas, em
relação a isso, devo ser perdoado. Meu amigo nunca dava explicações. Vai
ver ele me achava parecido com ele. Só que, infelizmente, eu não sei ver
carneiros dentro da caixa. Talvez eu me pareça um pouco com as pessoas
grandes. Devo ter envelhecido. \\

\textbf{V}

Todos os dias eu aprendia algo sobre o planeta, sobre a partida, sobre a
viagem. Isso vinha aos poucos, ao acaso de minhas reflexões. Foi assim
que, no terceiro dia, conheci o drama dos baobás.

Desta vez, foi graças ao carneiro, pois o pequeno príncipe me interrogou
subitamente, como se estivesse tomado por uma dúvida importante:

- Não é verdade que os carneiros comem arbustos?

- Sim, é verdade.

- Ah! Fico feliz em saber.

Não entendi por que era tão importante que os carneiros comem arbustos.
Mas o pequeno príncipe continuou:

- Sendo assim, eles também comem baobás?

Contei para o pequeno príncipe que os baobás não são arbustos, mas
árvores da altura de igrejas e que, mesmo que ele trouxesse uma tropa de
elefantes, não conseguiria dar fim num único baobá.

A ideia da tropa de elefantes o fez rir.

- É só colocá-los um em cima do outro...

E logo em seguida observou sabiamente:

- Antes de crescer, os baobás são pequenos.

- É claro! Mas por que você quer que os carneiros comam os pequenos
baobás?

Ele respondeu:

- Bom, vamos pensar! -- como se se tratasse de algo evidente.

Foi-me preciso um grande esforço intelectual para compreender sozinho o
problema.

Realmente, no planeta do pequeno príncipe, assim como em todos os
outros, existiam as ervas boas e as más. Portanto, existiam sementes
boas das ervas boas e sementes más das ervas más. Mas as sementes são
invisíveis. Dormem secretamente na terra até que uma delas receba a
graça de acordar. Então ela se espreguiça e lança a princípio
timidamente em direção ao sol um raminho inofensivo. Se for um ramo de
rabanete ou de roseira, pode deixar crescer à vontade. Mas, se for uma
planta má, é preciso arrancar a planta imediatamente, logo ao
reconhecê-la. Ora, existiam sementes terríveis no planeta do pequeno
príncipe... eram sementes de baobás. O chão do planeta estava cheio
delas. Se um baobá for descoberto tarde demais, não dá pra se livrar
dele. Ele obstrui o planeta inteiro. Perfura-o com suas raízes. E se o
planeta for pequeno demais, e se os baobás forem muitos, o planeta
explode.

- É uma questão de disciplina -- disse mais tarde o pequeno príncipe. --
Quando terminamos nossa toalete matinal devemos limpar cuidadosamente o
planeta. Temos que nos comprometer em arrancar sistematicamente os
baobás assim tão logo os distinguimos das roseiras, com as quais eles se
assemelham muito quando jovens. Trata-se de um trabalho entediante, mas
é bem fácil.

E um dia ele me aconselhou a fazer um belo desenho para fazer isso
entrar de uma vez por todas na cabeça das crianças do lugar onde eu
moro.

- Isso pode lhes ser útil se um dia forem viajar -- explicava ele. -- Às
vezes não há inconveniente em deixar um trabalho para mais tarde. Mas,
quando se trata de baobá, é sempre uma catástrofe. Conheci um planeta
habitado por um homem preguiçoso. Ele tinha ignorado três arbustos...

E, conforme as indicações do pequeno príncipe, desenhei o tal planeta.
Detesto assumir um tom moralista. Mas o perigo é tão pouco conhecido, e
os riscos corridos por aquele que se perde em um asteroide são de tal
forma consideráveis que, só desta vez, abro exceção para a minha
reserva. Falei:

- Crianças! Prestem atenção nos baobás!

Foi para avisar meus amigos sobre um perigo que há tanto tempo os
ameaçava, como a mim, sem que tomássemos conhecimento, que tanto me
dediquei àquele desenho. A lição que eu dava valia a pena. Talvez você
se pergunte: Por que neste livro não existem desenhos tão grandiosos
quanto os desenhos dos baobás? A resposta é muito simples: Eu tentei,
mas não consegui. Quando desenhei os baobás eu estava possuído pelo
sentimento de urgência.\\

\textbf{VI}

Ah, pequeno príncipe, pouco a pouco fui compreendendo sua vidinha
melancólica. Durante muito tempo você só teve como distração o sol se
pondo. Soube desse novo detalhe no quarto dia, pela manhã, quando você
disse:

- Eu adoro o pôr do sol. Vamos ver um...

- Mas temos que esperar...

- Esperar o quê?

- Esperar que o sol se ponha.

No começo você pareceu surpreso, mas depois riu de si mesmo. E disse:

- Eu sempre penso que estou em casa.

De fato. Como todos sabem, quando é meio-dia nos Estados Unidos, o sol
se põe na França. Bastaria chegar na França em um minuto para assistir
ao pôr do sol. É uma pena que a França fique tão longe. Mas, em seu
planetinha minúsculo, bastava mover a cadeira alguns passos. E poderia
contemplar o crepúsculo quantas vezes quisesse...

- Certo dia eu vi o sol se pôr 43 vezes!

E um pouco mais tarde acrescentou:

- Você sabe... quando estamos muito tristes adoramos o pôr do sol....

- Então no dia das 43 vezes você estava muito triste?

Mas o pequeno príncipe não respondeu.\\

\textbf{VII}

No quinto dia, graças ao carneiro, o segredo da vida do pequeno príncipe
foi-me revelado. Ele me perguntou de repente, sem preâmbulo, como se
fosse o fruto de um problema por muito tempo meditado em silêncio:

- Se um carneiro come arbustos, ele também come flores?

- Um carneiro come tudo o que encontra pela frente.

- Mesmo as flores com espinhos?

- Sim. Mesmo as com espinhos.

- Então para que servem os espinhos?

Eu não sabia. Estava muito ocupado tentando soltar um parafuso apertado
do meu motor. E começava a me preocupar porque a pane dava sinais de ser
grave, e a água potável que se esgotava me fazia temer o pior.

- Para que servem os espinhos?

O pequeno príncipe nunca renunciava a uma pergunta, uma vez que a tinha
proferido. Eu estava irritado com meu parafuso e respondi qualquer
coisa:

- Os espinhos não servem pra nada, é pura malvadeza das plantas.

- Oh!

Mas, passado um silêncio, ele me lançou, meio rancoroso:

- Não acredito em você! As flores são fracas. Ingênuas. Protegem-se como
podem. Acham-se horríveis com seus espinhos...

Não respondi. Nesse instante, pensei: ``Se esse parafuso continuar
assim, vou arrancá-lo com uma martelada''. Novamente o pequeno príncipe
atrapalhou meus pensamentos:

- Então você acha que as flores...

- Não, nada disso! Eu não acho nada! -- respondi a primeira coisa que me
veio à cabeça. -- Ocupo-me de coisas mais sérias!

Ele me olhou assombrado.

- Coisas mais sérias?

Via-me com o martelo, com os dedos pretos de graxa, debruçado sobre um
objeto que lhe parecia horrível.

- Você está falando como as pessoas grandes!

Senti-me um pouco envergonhado. Mas ele prosseguiu, impetuoso:

- Você está confundindo tudo... misturando tudo!

Ele ficou realmente irritado. Agitava ao vento os cabelos dourados:

- Conheço um planeta onde mora um senhor meio vermelho. Ele nunca
cheirou uma flor. Nunca olhou para uma estrela. Nunca amou ninguém. A
única coisa que ele fazia era contas. E, igual a você, ele fica
repetindo o dia inteiro,: ``Eu sou um homem sério! Eu sou um homem
sério!'' e isso o enchia de orgulho. Mas ele não é um homem, é um
cogumelo!

- Um o quê?

- Um cogumelo!

Agora o pequeno príncipe estava pálido de cólera.

- As flores produzem espinhos há milhões de anos. Os carneiros comem as
flores há milhões de anos. E não seria sério tentar compreender por que
elas sofrem tanto produzindo espinhos que nunca servem pra nada? Então
quer dizer que a guerra dos carneiros e das flores não é importante?
Isso não seria mais sério e mais importante do que as contas de um velho
gordo e vermelho? E se eu conhecesse uma flor única no mundo, que não
existe em nenhum outro lugar a não ser no meu planeta? Vamos supor que
um carneirinho pode dar-lhe fim numa única mordida, despretensiosamente,
numa manhã, sem se dar corta do que está fazendo... isso não seria
importante?

Ele enrubesceu e depois disse:

- Se uma pessoa gosta de uma flor que só tem um exemplar para milhares e
milhares de estrelas, isso basta para que fique feliz ao contemplá-la.
Ele vai pensar: ``Minha flor está lá, em algum lugar...'' Mas se por
acaso o carneiro come a flor é como se, pra ele, todas as estrelas se
apagassem bruscamente! E desde quando isso não é importante?

Ele não conseguiu dizer mais nada além disso. De repente rompeu em
soluços. A noite caiu. Deixei de lado meus instrumentos. Caçoava do meu
martelo, do meu parafuso, da sede e da morte. Havia, numa estrela, num
planeta, o meu, a Terra, um pequeno príncipe a ser consolado! Abracei-o.
Embalei-o. E lhe disse:

- A flor de que você tanto gosta não corre perigo... Vou desenhar uma
focinheira para o seu carneiro... E uma proteção para a sua flor...
Eu...

Eu não sabia muito bem o que dizer. Sentia-me muito desajeitado. Não
sabia como esperá-lo, onde encontrá-lo... É tão misterioso o país das
lágrimas!

}
%\ParallelPar

\end{Parallel}











\chapter{VIII}

Bem rápido, aprendi mais sobre aquela flor. No planeta do pequeno
príncipe, sempre existiram flores muito simples, decoradas com uma única
fileira de pétalas que não ocupavam demasiado espaço e que não
atrapalhavam ninguém. Elas apareciam na relva pela manhã e apagavam-se à
noite. Mas aquela tinha germinado um dia, de um grão trazido sabe-se lá
de onde, e o pequeno príncipe vigiou de muito perto esse ramo que nada
tinha a ver com os outros. Talvez fosse um tipo de baobá. Mas logo o
arbusto parou de crescer e começou a esboçar uma flor. O pequeno
príncipe, que assistia à instalação de um botão enorme, percebeu que
seria uma aparição milagrosa, mas a flor não parava de se preparar para
ficar bonita, sob o abrigo de seu quarto verde. Ela escolhia
cuidadosamente suas cores. Não queria sair amassada como as papoulas. E
só pretendia aparecer no auge de sua beleza. Ah, com certeza! Era
vaidosa! Sua misteriosa toalete durou dias e dias. E eis que, certa
manhã, exatamente na hora do nascer do sol, ela deu as caras.

E ela, que havia trabalhado com tanta precisão, disse, bocejando:

- Ah! Estou acabando de acordar... Peço-lhe desculpas... Ainda estou
toda despenteada...

O pequeno príncipe não pôde conter sua admiração:

- Como você é linda!

- Não é mesmo? -- respondeu a flor. -- E nasci ao mesmo tempo que o
sol...

O pequeno príncipe percebeu que ela não era tão modesta, mas era
encantadora!

- Acho que está na hora do café da manhã -- ela continuou --, você teria
a bondade de pensar em mim...

E o pequeno príncipe, desconcertado, foi pegar um regador com água
fresca para servir à flor.

Com isso ela logo o incomodou com sua vaidade melindrosa. Certo dia, por
exemplo, ao falar de seus quatro espinhos, ela disse ao pequeno
príncipe:

- Podem vir os tigres com suas garras!

- Não existem tigres no meu planeta -- rebateu o pequeno príncipe --, e
além disso os tigres não comem ervas.

- Eu não sou uma erva -- respondeu delicadamente a flor.

- Me desculpe...

- Eu não nenhum tenho medo de tigres, mas tenho horror a correntes de
ar. Por acaso você não teria um pára-vento?

- Horror das correntes de ar... que azar para uma planta -- ponderou o
pequeno príncipe. -- Essa flor é bem complicada...

- À noite você me colocará sob a redoma. Faz muito frio aqui no seu
país. E as instalações não são as melhores... Lá de onde eu venho, por
exemplo...

Mas aí ela se interrompeu. Viera em forma de semente. Não pôde conhecer
nada de outros mundos. Humilhada por ter sido surpreendida preparando
uma mentira tão ingênua quanto aquela, tossiu umas duas ou três vezes
antes de conturbar o pequeno príncipe:

- E aquele pára-vento de que falávamos?...

- Eu ia buscá-lo mas você estava conversando comigo!

Então ela forçou sua tosse para provocar-lhe remorsos.

Assim, o pequeno príncipe, apesar da boa vontade de seu amor, logo
começou a desconfiar da flor. Tomou a sério palavras sem importância e
se entristeceu.

- Eu não deveria ter dado ouvidos a ela -- confiou-me ele um dia --,
nunca se deve dar ouvido às flores. Apenas olhá-las e cheirá-las. A
minha perfumava meu planeta, mas eu não sabia me satisfazer com ela.
Essa história de garras, que tanto me incomodou, deveria ter me
enternecido...

Ele também me confiou:

- Eu não soube compreender aquilo! Eu deveria ter julgado seus atos e
não suas palavras. Ela me perfumava e me iluminava. Eu nunca deveria ter
fugido! Eu deveria ter adivinhado sua ternura por trás dos pobres
artifícios. As flores são tão contraditórias! Mas eu era muito jovem
para saber amá-la.

\chapter{IX}

Para sua fuga, creio que ele aproveitou de uma migração de pássaros
selvagens. Na manhã de partida, pôs em ordem seu planeta. Retirou
cuidadosamente as cinzas do interior dos vulcões em atividade. Ele
possuía dois vulcões em atividade. Era muito cômodo para esquentar o
café da manhã. Possuía também um vulcão apagado. Mas, como ele mesmo
dizia: ``A gente nunca sabe o que vai acontecer!''. Também desobstruiu o
vulcão apagado. Se estiverem bem desobstruídos, os vulcões queimam lenta
e regularmente, sem erupções. As erupções vulcânicas são como fogos de
chaminé. Evidentemente, em nossa terra, somos pequenos demais para
limpar nossos vulcões. É por isso que eles nos preocupam tanto.

O pequeno príncipe também arrancou, meio melancólico, os últimos brotos
de baobá. Ele julgava nunca mais ter que voltar. Mas, naquela manhã,
todos aqueles trabalhos familiares lhe pareceram extremamente
agradáveis. E quando regou a flor pela última vez e se preparou para
protegê-la dentro da redoma, percebeu que sentia vontade de chorar.

- Adeus -- disse ele à flor.

Mas ela não lhe respondeu.

- Adeus -- repetiu.

A flor tossiu. Mas não por conta do resfriado.

- Fui uma tola -- disse ela, finalmente. -- Peço desculpas. Trate de ser
feliz.

Ele se surpreendeu com a falta de repreensão. Continuou ali,
desconcertado, segurando a redoma no ar. Não entendia aquela calma
amabilidade.

- Saiba que eu te amo -- disse-lhe a flor. -- Você nunca soube disso,
por minha culpa. Mas não há problema. Você foi tão tolo quanto eu. Trate
de ser feliz... Deixe essa redoma tranquila. Eu não a quero mais.

- Mas e o vento...

- Eu não estou tão gripada assim... O ar fresco da noite vai me fazer
bem. Sou uma flor.

- Mas e os animais...

- Devo suportar duas ou três lagartas se eu quiser conhecer as
borboletas. Dizem que são tão bonitas. Senão, quem virá me visitar? Você
vai estar longe. E, quanto aos animais grandes, não tenho nenhum medo.
Tenho minhas garras.

E ela mostrou ingenuamente seus quatro espinhos. Depois acrescentou:

- Não fique aí andando de um lado para o outro, isso me irrita. Você
decidiu ir embora. Então vá.

Ela não queria que ele a visse chorar. Era uma flor tão orgulhosa...

\chapter{X}

O pequeno príncipe se encontrava na região dos asteroides 325, 327, 328,
329 e 330. Ele então começou visitando cada um deles para procurar uma
ocupação e para se instruir.

O primeiro era habitado por um rei. O rei, vestido de púrpura e de
arminho, sentava-se num trono muito simples, porém majestoso.

- Ah! Eis um súdito! -- exclamou o rei ao avistar o pequeno príncipe. E
o pequeno príncipe se perguntou:

- Como ele pode me reconhecer se nunca me viu antes?

Ele não sabia que, para os reis, o mundo é muito simplificado. Todos os
homens são súditos.

- Aproxime-se para eu te ver melhor -- disse o rei, cheio de orgulho por
reinar sobre alguém.

O pequeno príncipe olhou à sua volta, procurando onde se sentar, mas o
planeta estava obstruído pelo magnífico manto de arminho. Então ficou de
pé, e, como estava cansado, bocejou.

- É contrário à etiqueta bocejar na presença de um rei -- disse-lhe o
monarca. -- Eu proíbo que você faça isso.

- Eu não posso evitar -- respondeu o príncipe, confuso. -- Fiz uma longa
viagem e não dormi.

- Então -- disse o rei -- eu lhe ordeno que boceje. Faz anos que eu não
vejo alguém bocejar. Os bocejos, para mim, têm uma singularidade. Vamos!
Continue bocejando. É uma ordem.

- Isso me intimida... eu não consigo mais... -- respondeu o pequeno
príncipe, enrubescendo.

- Hum! Hum! -- respondeu o rei. -- Então... eu te ordeno intercalar
bocejo e...

Ele balbuciava um pouco e parecia vexado.

Pois o rei fazia questão que sua autoridade fosse respeitada. Não
tolerava a desobediência. Era um monarca absoluto. Mas, como era muito
bom, suas ordens eram razoáveis.

- Se eu ordenasse -- costumava dizer --, se eu ordenasse a um general
para que se transformasse em um pássaro do oceano, e se o general não
obedecesse, a culpa não seria do general. Seria minha.

- Posso me sentar? -- interrogou timidamente o pequeno príncipe.

- Eu lhe ordeno que sente -- respondeu o rei, puxando majestosamente um
pedaço do manto de arminho.

Mas o pequeno príncipe estava espantado. O planeta era minúsculo. Sobre
quem o rei poderia reinar?

- Senhor... -- disse ele --, peço desculpas por interrogar-lhe...

- Eu ordeno que me interrogue -- apressou-se em dizer o rei.

- Senhor... sobre o que você reina?

- Sobre tudo -- respondeu o rei, com uma grande simplicidade.

- Sobre tudo?

O rei, com um gesto discreto, indicou seu planeta, os outros planetas e
as estrelas.

- Sobre tudo isso? -- inquiriu o pequeno príncipe.

- Sobre tudo isso... -- respondeu o rei.

Pois não bastava ele ser um monarca absoluto, ele era um monarca
universal.

- E as estrelas te obedecem?

- É claro que sim -- disse-lhe o rei. -- Elas obedecem imediatamente. Eu
não tolero a falta de disciplina.

Um poder como aquele deixou o príncipe maravilhado. Se ele próprio fosse
o detentor, teria podido assistir não a 44, mas a 70, e até 100, ou quem
sabe a 200 pores de sol no mesmo dia, sem precisar mudar a cadeira do
lugar! E como se sentia meio triste devido à lembrança de seu planetinha
abandonado, ousou solicitar uma graça do rei:

- Eu gostaria de ver um pôr do sol... Dá-me esse prazer... Ordene que o
sol se ponha...

- Se eu ordenasse que um general voasse de uma flor a outra como faz uma
borboleta, ou que ele escrevesse uma tragédia, ou que se transformasse
numa ave marinha, e se esse general não executasse a ordem recebida quem
estaria errado, ele ou eu?

- Você -- disse o pequeno príncipe, convicto.

- Exato. É preciso exigir de cada um o que cada um pode dar -- retorquiu
o rei. -- A princípio, a autoridade está na razão. Se você ordenar que
seu povo se atire na água, o povo fará uma revolução. Tenho o direito de
exigir a obediência porque minhas ordens são razoáveis.

- Pois bem, mas e o meu pôr do sol? -- lembrou o pequeno príncipe, que
nunca esquecia uma pergunta, uma vez que a tinha proferido.

- Você terá o seu pôr do sol. Exigirei isso. Mas, de acordo com meu
conhecimento de governo, esperarei que as condições sejam favoráveis.

- Quando vai acontecer? -- queria saber o pequeno príncipe.

- Hein? Hein? -- respondeu o rei, consultando um grande calendário. --
Hein? Hein? Vai ser por volta de... de... de 7:40! E você vai ver como
sou bem obedecido.

O pequeno príncipe bocejou. Ele se lamentava pelo pôr do sol que havia
perdido. E já começava a se entediar:

- Eu não tenho mais o que fazer aqui -- disse ao rei. -- Vou embora!

- Não vá -- respondeu o rei, que estava orgulhoso por possuir um súdito.
-- Não vá embora, eu lhe farei ministro!

- Ministro de quê?

- Da... justiça!

- Mas não há ninguém para julgar!

- Não dá pra saber -- disse o rei. -- Eu ainda não percorri todo o meu
reinado. Estou muito velho, não tenho espaço para uma carruagem e me
canso de andar.

- Oh! Mas eu já vi -- disse o pequeno príncipe, debruçando-se para dar
mais uma olhada do outro lado do planeta. -- Também não há ninguém
ali...

- Então julgarás a si mesmo -- respondeu-lhe o rei. -- É o mais difícil.
É muito mais difícil julgar-se a si mesmo do que o outro. Se você for
capaz de se julgar bem é porque é um verdadeiro sábio.

- Eu -- disse o pequeno príncipe --, eu posso julgar a mim mesmo em
qualquer lugar. Não preciso viver aqui.

- Hein? Hein? -- disse o rei. -- Acho que em algum lugar no meu planeta
vive um rato. Eu o ouço à noite. Você poderá julgar esse rato velho. Irá
condená-lo à morte de tempos em tempos. Assim, a vida dele depende da
sua justiça. Mas irá perdoá-lo a cada vez para poupá-lo. Pois não há
outro.

- Não gosto de condenar à morte -- respondeu o pequeno príncipe --, acho
que vou embora.

- Não -- disse o rei.

Mas o pequeno príncipe, tendo terminado seus preparativos, não quis
magoar o velho monarca:

- Se vossa Majestade quisesse ser prontamente obedecida, poderia dar-me
uma ordem razoável. Poderia me ordenar, por exemplo, para ir embora em
menos de um minuto. As condições me parecem favoráveis...

Como o rei não tinha o que responder, o pequeno príncipe hesitou a
princípio, em seguida suspirou e partiu...

- Eu te faço meu embaixador -- apressou-se em gritar o rei.

Tinha um ar de grande superioridade.

- As pessoas grandes são muito estranhas -- pensou o pequeno príncipe,
durante sua viagem.

\chapter{XI}

O segundo planeta era habitado por um vaidoso:

- Ah! Eis um admirador que vem de visita! -- exclamou de longe o vaidoso
ao avistar o pequeno príncipe.

Pois, para os vaidosos, os outros homens não passam de admiradores.

- Bom dia -- disse o pequeno príncipe. -- Você tem um chapéu engraçado.

- É para agradecer -- respondeu-lhe o vaidoso. -- É para agradecer
quando me aclamam. Infelizmente, nunca ninguém passa por aqui.

- Ah, é? -- disse o pequeno príncipe, sem entender.

- Bata suas mãos uma na outra -- aconselhou o vaidoso.

O pequeno príncipe bateu as mãos uma na outra. O vaidoso agradeceu
modestamente levantando o chapéu.

``Isso é mais divertido do que a visita ao rei'', pensou o pequeno
príncipe. E bateu novamente as mãos. O vaidoso voltou a cumprimentar
levantando o chapéu.

Passados cinco minutos de exercício, o pequeno príncipe se cansou da
monótona brincadeira e perguntou:

- E o que é preciso fazer para que o chapéu caia?

Mas o vaidoso não ouviu. Os vaidosos nunca ouvem nada que não seja
elogio.

- É verdade que você me admira muito?

- O que significa ``admirar''?

- ``Admirar'' significa reconhecer que eu sou o homem mais bonito, o
mais bem vestido, o mais rico e o mais inteligente do planeta.

- Mas você mora sozinho em seu planeta!

- Dá-me esse prazer. Admire-me mesmo assim!

- Eu te admiro -- respondeu o pequeno príncipe, fazendo um sinal de
indiferença --, mas em que isso pode te interessar?

E o pequeno príncipe partiu.

``As grandes pessoas são mesmo muito esquisitas'', pensou durante a
viagem.

\chapter{XII}

O planeta seguinte era habitado por um bêbado. Essa visita foi muito
curta, mas ela mergulhou o pequeno príncipe numa profunda melancolia:

- O que você faz aqui? -- perguntou ao bêbado, que estava em silêncio
diante de uma coleção de garrafas vazias e outra de garrafas cheias.

- Eu bebo -- respondeu o bêbado, com um ar lúgubre.

- Por que você bebe? -- perguntou o pequeno príncipe.

- Para esquecer -- respondeu o bêbado.

- Para esquecer de quê? -- inquiriu o pequeno príncipe, que já começava
a ter dó dele.

- Para esquecer que tenho vergonha -- confessou o bêbado, baixando a
cabeça.

- Vergonha de quê? -- quis saber o pequeno príncipe, desejando
socorrê-lo.

- Vergonha de beber! -- concluiu o bêbado, fechando-se definitivamente
em seu silêncio.

E o pequeno príncipe partiu, perplexo.

``As pessoas grandes são mesmo muito muito esquisitas'', pensou ele,
durante a viagem.

\chapter{XIII}

O quarto planeta era o do homem de negócios. Este homem estava tão
ocupado que nem mesmo levantou a cabeça para a chegada do pequeno
príncipe.

- Bom dia -- disse-lhe este. -- Seu cigarro está apagado.

- Três e dois são cinco. Cinco mais sete, doze. Doze mais três, quinze.
Bom dia. Quinze mais sete, vinte e dois. Vinte e dois mais seis, vinte e
oito. Não tenho tempo de acendê-lo. Vinte e seis mais cinco, trinta e
um. Ufa! Isso dá quinhentos e um milhões, seiscentos e vinte e dois mil,
setecentos e trinta e um.

- Quinhentos milhões de quê?

- Hein? Você ainda está aí? Quinhentos e um milhões de... esqueci...
tenho tanto trabalho! Sou um cara muito sério, não perco tempo com
bobagens! Dois mais cinco, sete...

- Quinhentos milhões de quê? -- repetiu o pequeno príncipe, que nunca em
sua vida havia renunciado uma pergunta, uma vez que a tinha proferido.

O homem de negócios levantou a cabeça:

- Faz 54 anos que eu vivo neste planeta e só fui interrompido três
vezes. A primeira vez foi há 22 anos, por um besouro que caiu aqui
sabe-se deus de onde. Ele fazia um barulho horrível e por causa dele
errei quatro vezes uma soma. A segunda vez foi há 11 anos, por causa de
uma crise de reumatismo. Sinto falta de exercício. Não tenho tempo para
passear. Sou um homem sério. A terceira vez... ahá! Como eu dizia,
quinhentos e um milhões...

- Milhões de quê?

O homem de negócios entendeu que não havia qualquer esperança de paz:

- Dessas coisinhas que às vezes a gente vê no céu.

- Moscas?

- Não, essas coisinhas que brilham.

- Abelhas?

- Não! Essas coisinhas douradas que fazem os ociosos sonharem. Mas eu
sou um homem sério! Não tenho tempo para devaneios.

- Ah! As estrelas?

- Exatamente. As estrelas.

- E o que você faz com quinhentos milhões de estrelas?

- Quinhentos e um milhões, cento e vinte e duas mil, setecentos e trinta
e uma. Sou um homem sério, sou muito preciso.

- E o que você faz com essas estrelas?

- O que eu faço com elas?

- Sim.

- Nada. Eu as possuo.

- Você possui as estrelas?

- Sim.

- Mas eu já vi um rei que...

- Os reis não a possuem. Eles ``reinam'' sobre elas. É muito diferente.

- E para que serve possuir as estrelas?

- Para me deixar rico.

- E para que serve ficar rico?

- Para comprar outras estrelas, se alguém as encontrar.

``Esse daí, pensou o pequeno príncipe, raciocina como aquele bêbado.''

No entanto, continuou perguntando:

- Como é possível possuir as estrelas?

- De quem são elas? -- revidou, ameaçador, o homem de negócios.

- Eu não sei. De ninguém.

- Então elas são minhas porque eu pensei nisso primeiro.

- E isso basta?

- É claro que sim. Quando você encontra um diamante que não pertence a
ninguém, ele passa a ser seu. Quando você tem uma ideia em primeiro
lugar, você a registra em seu nome: ela é sua. E, nesse caso, eu possuo
as estrelas, porque ninguém antes de mim imaginou possuí-las.

- É verdade -- disse o pequeno príncipe. -- E o que você faz com elas?

- Eu as administro. Eu as conto e as conto novamente -- disse o homem de
negócios. -- É muito difícil. Mas eu sou um homem sério!

O pequeno príncipe não se dava por satisfeito.

- Se eu tenho um cachecol, posso colocá-lo em volta do meu pescoço e
andar com ele. Se eu tenho uma flor, posso colhê-la e levá-la comigo.
Mas você não pode colher as estrelas!

- Não, mas posso colocá-las no banco.

- O que você quer dizer com isso?

- Isso quer dizer que eu escrevo o número das minhas estrelas num
papelzinho. E depois tranco esse papel a chaves numa gaveta.

- Só isso?

- Isso é o suficiente!

``É divertido'', pensou o pequeno príncipe. ``É muito poético. Mas não é
muito sério.''

Em se tratando de coisas sérias, o pequeno príncipe tinha ideias muito
diferentes em relação às pessoas grandes.

- Tenho uma flor e a rego diariamente -- continuou ele. -- Tenho três
vulcões dos quais limpo as chaminés semanalmente. Porque eu também limpo
o que está extinto. Nós nunca sabemos o que pode acontecer. É útil aos
meus vulcões, é útil à minha flor que eu as possua. Mas você não é útil
para as estrelas...

O homem de negócios abriu a boca mas não encontrou nenhuma resposta, e o
pequeno príncipe partiu.

``As pessoas grandes são realmente extraordinárias, pensava ele durante
a viagem.''

\chapter{XIV}

O quinto planeta era muito curioso. Era o menor de todos. Quase não
tinha espaço para um lampião e um acendedor de lampião. O pequeno
príncipe não conseguia entender para que poderia servir, em algum lugar
no céu, num planeta sem casa nem população, um lampião e um acendedor de
lampião. Entretanto, pensou:

``Talvez esse homem seja mesmo absurdo. No entanto, ele é menos absurdo
do que o rei, do que o vaidoso, do que o homem de negócios e o bêbado.
Pelo menos seu trabalho tem um sentido. Quando ele acende o lampião, é
como se desse à luz a mais uma estrela, ou a uma flor. Quando ele apaga
o lampião, a flor ou a estrela dormem. É um uma bonita ocupação. E é
útil por ser bonita''.

Assim que abordou o planeta, cumprimentou respeitosamente o acendedor:

- Bom dia. Por que você está apagando o lampião?

- É a regra -- respondeu o acendedor. -- Bom dia!

- E qual é a regra?

- Devo apagar meu lampião. Boa noite.

E o acendeu novamente

- Mas por que agora você está acendendo?

- É a regra -- respondeu o acendedor.

- Não estou entendendo -- disse o pequeno príncipe.

- Não há o que entender -- disse o acendedor. -- Regra á regra. Bom dia.

E apagou o lampião.

Depois ele limpou a testa com um lenço xadrez vermelho.

- Executo uma tarefa terrível. Antes fazia sentido. Eu apagava de manhã
e acendia à noite. Eu tinha o resto do dia para descansar e o resto da
noite para dormir...

- E desde então a regra mudou?

- A regra não mudou -- disse o acendedor. -- É aí que mora o drama!
Entra ano, sai ano, o planeta gira cada vez mais rápido, mas a regra não
mudou!

- Então o que houve? -- perguntou o pequeno príncipe?

- Agora que ela dá uma volta por minuto, eu não tenho mais nem um minuto
de descanso. Acendo e apago uma vez a cada minuto!

- Que engraçado! Os dias no seu planeta duram um minuto!

- Não tem graça nenhuma -- disse o acendedor. -- Já faz um mês que
estamos conversando.

- Um mês?

- Sim. Trinta minutos. Trinta dias! Boa noite.

E ele acendeu outra vez seu lampião.

O pequeno príncipe olhou pra ele e sentiu que amava aquele acendedor tão
fiel à regra. Lembrou-se dos pores do sol que antigamente procurava,
mudando a cadeira de lugar. Ele quis ajudar seu amigo:

- Sabe, eu tenho uma ideia para que possa descansar quando quiser...

- Eu sempre quero descansar -- respondeu o acendedor.

Pois é possível ser fiel e preguiçoso ao mesmo tempo.

O pequeno príncipe prosseguiu:

- Seu planeta é tão pequeno que você pode dar a volta nele com três
passadas. Você só precisa andar bem devagar para continuar no chão.
Quando quiser descansar, é só andar... e assim o dia vai durar o tempo
que você quiser.

- Isso não me ajuda muito -- disse o acendedor. -- O que eu mais gosto
de fazer na vida é dormir.

- Ah, então não tem nada a ser feito -- disse o pequeno príncipe.

- Não tem mesmo -- disse o acendedor. -- Bom dia!

E ele apagou o lampião.

``Esse daí, pensou o pequeno príncipe'', enquanto continuava sua viagem
para cada vez mais longe ``esse daí será desprezado por todos os outros,
pelo rei, pelo vaidoso, pelo bêbado, pelo homem de negócios. No entanto,
é o único que não me parece ridículo. Talvez isso aconteça porque ele se
ocupa de outra coisa além de si mesmo.''

Deu um longo suspiro de arrependimento e continuou pensando:

``Esse é o único que poderia virar meu amigo. Mas seu planeta é mesmo
muito pequeno. Não tem lugar para dois...''

O que o pequeno príncipe não ousava admitir é que ele já sentia falta
desse planeta abençoado sobretudo por causa dos mil quatrocentos e
quarenta pores do sol durante vinte e quatro horas!

\chapter{XV}

O sexto planeta era um planeta dez vezes mais amplo. Era habitado por um
velho senhor que escrevia livros enormes.

- Veja só! Eis um explorador! -- exclamou ele ao notar o pequeno
príncipe.

O pequeno príncipe sentou-se na mesa, ofegante. Já havia viajado tanto!

- De onde você vem? -- perguntou o velho senhor.

- Que livro tão grosso é esse? -- perguntou o pequeno príncipe. -- O que
você faz aqui?

- Eu sou geógrafo -- respondeu o velho senhor.

- O que é um geógrafo?

- É um estudioso que sabe onde estão os mares, os rios, as cidades, as
montanhas e os desertos.

- Isso é muito interessante -- respondeu o pequeno príncipe. -- Essa sim
é uma verdadeira profissão!

E passou os olhos em volta, observando o planeta do geógrafo. Nunca
tinha visto um planeta tão majestoso.

- Seu planeta é lindo. Existem oceanos por aqui?

- Não sei -- respondeu o geógrafo.

- Ah! (O pequeno príncipe ficou decepcionado.) E montanhas?

- Não sei -- respondeu o geógrafo.

- E cidades, rios, desertos?

- Também não sei -- respondeu o geógrafo.

- Mas você é geógrafo!

- É verdade -- disse o geógrafo --, mas não sou explorador. Sinto muita
falta de exploradores. Não é o geógrafo que vai fazer a conta das
cidades, dos rios, das montanhas, dos mares, dos oceanos e dos desertos.
O geógrafo é muito importante para passear. Ele não sai do escritório.
Mas recebe os exploradores. Ele os interroga, toma nota de suas
lembranças. E se as lembranças de um deles lhe parecer interessante, faz
uma enquete sobre a moral do explorador.

- Por quê?

- Porque um explorador que mentisse provocaria catástrofes nos livros de
geografia. Assim como um explorador que bebesse demais.

- Por quê? -- indagou o pequeno príncipe.

- Porque os bêbados veem duplamente. Então o geógrafo registraria duas
montanhas onde só existe uma.

- Conheço alguém -- disse o pequeno príncipe -- que seria um péssimo
explorador.

- É possível. Então, quando a moral do explorador parece boa, faz-se uma
enquete sobre sua descoberta.

- Vão verificar?

- Não. É muito complicado. Mas exigimos que o explorador forneça provas.
Se por exemplo, se tratar da descoberta de uma grande montanha, exigimos
que ele traga pedras grandes.

De repente o geógrafo se calou.

- Mas você está vindo de longe! É um explorador! Você vai me descrever
seu planeta!

E o geógrafo, tendo aberto seu registro, fez uma marca a lápis. A
princípio, anota-se a lápis as histórias dos exploradores. Antes de
anotar a caneta, espera-se que o explorador forneça provas.

- E então? -- interrogou o geógrafo.

- Oh! O meu país -- diz o pequeno príncipe -- não é muito interessante,
ele é pequeno demais. Eu tenho três vulcões. Dois vulcões em atividade e
um vulcão extinto. Mas nunca se sabe.

- Não, nunca se sabe -- disse o geógrafo.

- Eu também tenho uma flor.

- Nós não registramos as flores -- disse o geógrafo.

- Por que não? É o mais bonito!

- Por que as flores são efêmeras.

- O que significa ``efêmera''?

- As geografias -- respondeu o geógrafo -- são os livros mais sérios que
existem. Nunca saem de moda. É raro que uma montanha se mova de lugar. É
raro que um oceano se esvazie por completo. Nós escrevemos coisas
eternas.

- Mas os vulcões apagados podem acordar -- interrompeu o pequeno
príncipe. -- O que significa ``efêmero''?

- Estejam os vulcões extintos ou em atividade, dá no mesmo pra nós --
disse o geógrafo. -- O que conta é a montanha. Ela não muda.

- Mas o que significa ``efêmero''? -- repetiu o pequeno príncipe que, em
sua vida, nunca havia renunciado a uma pergunta, um vez que a tinha
proferido.

- Significa ``ameaçada de eminente desaparição''.

- Minha flor pode morrer logo?

- É claro que sim.

``Minha flor é efêmera'', pensou o pequeno príncipe, ``ela tem apenas
quatro espinhos para se defender contra o mundo! E a deixei sozinha em
casa!''

Esse foi seu primeiro movimento de arrependimento. Mas ele tomou
coragem:

- O que você me aconselha a visitar? -- perguntou.

- O planeta Terra -- respondeu-lhe o geógrafo. -- Ele tem uma boa
reputação...

E o pequeno príncipe partiu, pensando em sua flor.

\chapter{XVI}

O sétimo planeta foi então a Terra.

A Terra não é um planeta qualquer! Conta-se 111 reis (sem esquecer, é
claro, dos 3 reis negros), 7 mil geógrafos, 900 mil homens de negócio, 7
milhões e meio de bêbados, 311 milhões de vaidosos. Ou seja, mais ou
menos 2 bilhões de pessoas grandes.

Para lhes dar uma ideia das dimensões da Terra, eu diria que antes da
invenção da eletricidade era preciso manter, no total de seis
continentes, um verdadeiro exército de 462.511 mil acendedores de
lampiões.

Visto com uma certa distância, isso criava um efeito espetacular. Os
movimentos desse exército eram regrados como os de um balé de ópera.
Primeiro vinha a vez dos acendedores dos postes da Nova Zelândia e da
Austrália. Depois os que, tendo acendido seus lampiões, iam dormir.
Então era a vez da dança dos acendedores de lampiões da China e da
Sibéria. Depois eles também se retiravam para os bastidores. Então vinha
a vez dos acendedores de postes da Rússia e das Índias. Depois os da
África e da Europa. Depois os da América do Sul. Depois os da América do
Norte. E eles nunca se confundiam na ordem de entrar em cena. Era
grandioso.

Sozinhos, o acendedor do único lampião do polo Norte e seu camarada do
único lampião do polo Sul levavam uma vida de ócio e preguiça:
trabalhavam duas vezes por ano.

\chapter{XVII}

Quando nossa intenção é provocar graça, às vezes mentimos um pouco. Eu
não fui muito honesto quando contei sobre os acendedores de lampiões.
Corro o risco de dar uma falsa ideia do nosso planeta àqueles que não o
conhecem. Os homens ocupam muito pouco lugar na Terra. Se os 2 bilhões
de habitantes que vivem na Terra ficassem de pé e um pouco apertados,
como se estivessem reunidos, ocupariam facilmente uma praça pública de
20 mil milhas de comprimento por 20 mil milhas de largura. Seria
possível amontoar a humanidade em uma pequena ilha do Pacífico.

As pessoas grandes, é claro, não vão acreditar em você. Elas acham que
ocupam muito espaço. Creem-se importantes como os baobás. Você pode lhes
sugerir que façam o cálculo. Elas adoram os números: isso as deixará
felizes. Mas não perca seu tempo com esse exercício. É inútil. Vocês
acreditam em mim.

O pequeno príncipe, ao chegar na Terra, ficou muito surpreso em não
encontrar ninguém. Ele já começava a temer ter-se enganado de planeta,
quando uma anel da cor da lua se moveu na areia.

- Boa noite -- disse o pequeno príncipe despretensiosamente.

- Boa noite -- respondeu a serpente.

- Em que planeta estou? -- perguntou o pequeno príncipe.

- Na Terra, na África -- respondeu ela.

- Ah!... Então não tem ninguém na Terra?

- Aqui é o deserto. Ninguém fica nos desertos. A Terra é grande -- disse
a serpente.

O pequeno príncipe se sentou numa pedra e olhou para o céu:

- Fico pensando -- disse ele -- se as estrelas são iluminadas para que,
um dia, cada um possa encontrar a sua. Veja o meu planeta. Ele está logo
abaixo de nós... Mas veja como está longe!

- Ele é bonito -- disse a serpente. -- O que você veio fazer aqui?

- Tenho problemas com uma flor -- disse o pequeno príncipe.

- Ah! -- exclamou a serpente.

E se calaram.

- Onde estão os homens? -- retomou finalmente o pequeno príncipe. --
Estamos meio sozinhos aqui no deserto.

- Também ficamos sozinhos onde estão os homens -- respondeu a serpente.

O pequeno príncipe a olhou durante um tempo:

- Você é um animal muito engraçado -- disse-lhe, enfim --, magra como um
dedo...

- Mas eu sou mais poderosa do que o dedo de um rei. -- disse a serpente.

O pequeno príncipe sorriu:

- Você não é tão poderosa... você nem tem patas... nem consegue
viajar...

- Eu posso te levar muito mais longe do que um navio -- disse a
serpente.

Enrolou-se no tornozelo do pequeno príncipe, como um pulseira de ouro:

- Aquele que eu tocar, devolvo-o à terra de onde saiu -- acrescentou. --
Mas você é puro e vem de uma estrela...

O pequeno príncipe não respondeu.

- Você me dá pena, é tão fraquinho perto dessa Terra de granito. Eu
posso te ajudar, um dia, se você estiver com muitas saudades do seu
planeta. Eu posso...

- Ah! Entendi -- disse o pequeno príncipe --, mas por que você fala
sempre por enigmas?

- Eu resolvo todos eles -- respondeu a serpente.

E se calaram.

\chapter{XVIII}

O pequeno príncipe atravessou o deserto e encontrou apenas uma flor. Uma
flor de três pétalas, uma florzinha de nada...

- Bom dia -- disse o pequeno príncipe.

- Bom dia -- respondeu a flor.

- Onde estão os homens? -- perguntou ele, educadamente.

A flor, certo dia, tinha visto passar uma caravana:

- Os homens? Acho que só existem uns seis ou sete. Eu os vi há alguns
anos. Mas nunca sabemos onde encontrá-los. O vento os leva. Eles não têm
raízes, elas o incomodam.

- Adeus -- disse o pequeno príncipe.

- Adeus -- respondeu a flor.

\chapter{XIX}

O pequeno príncipe escalou uma alta montanha. As únicas montanhas que
conhecera eram os três vulcões que batiam em seu joelho. E ele usava o
vulcão extinto como se fosse um banquinho.

``De uma montanha tão alta como esta'', pensou, ``vou poder ver todo o
planeta e todos os homens...''

Mas ele só avistava as pontas afiadas das rochas.

- Bom dia -- arriscou.

- Bom dia... Bom dia... Bom dia... -- respondeu o eco.

- Quem é você? -- perguntou o pequeno príncipe.

- Quem é você... quem é você... quem é você... -- respondeu o eco.

- Seja meu amigo, estou sozinho -- disse ele.

- Estou sozinho... estou sozinho... estou sozinho... -- respondeu o eco.

- Que planeta esquisito! -- pensou. -- Ele está completamente seco,
pontudo e salgado. E os homens não têm imaginação... No meu país eu
tinha uma flor: ela sempre falava primeiro...

\chapter{XX}

Mas aconteceu que o príncipe, tendo andando muito pelas areias, pelas
rochas e pelas neves, finalmente descobriu uma estrada. E todas as
estradas levam aos homens.

- Bom dia -- disse ele.

Era um jardim cheio de rosas.

- Bom dia -- responderam as rosas.

O pequeno príncipe olhou pra elas. Todas se pareciam com a sua flor.

- Quem são vocês? -- ele perguntou, atordoado.

- Nós somos rosas -- responderam as rosas.

- Ah! -- disse o pequeno príncipe.

E ele se entristeceu. Sua flor lhe dissera que ela era a única de sua
espécie no universo. E ali havia 5 mil, todas parecidas, num único
jardim!

``Ela ficaria muito incomodada'', pensou, ``se visse isso... ia tossir e
fingiria morrer para escapar do ridículo. E eu seria obrigado a fingir
que estava cuidando dela, porque, senão, para humilhar também a mim, ela
se deixaria morrer de verdade...

Depois ele ainda pensou:

``Eu me acreditava rico por ter uma única flor, e o que tenho é apenas
uma flor comum. Ela, e meus três vulcões que dão nos meus joelhos, e dos
quais um deles, talvez, esteja extinto para sempre, isso não faz de mim
um grande príncipe...''

E, deitado na grama, chorou.

\chapter{XXI}

Foi então que a raposa apareceu:

- Bom dia -- disse a raposa.

- Bom dia -- respondeu educadamente o pequeno príncipe, voltando-se, mas
sem encontrar nada.

- Estou aqui -- disse a voz, sob a macieira...

- Quem é você? -- perguntou o pequeno príncipe. -- Você é tão linda...

- Sou uma raposa -- disse a raposa.

- Venha brincar comigo -- propôs o pequeno príncipe. -- Eu estou tão
triste...

- Não posso brincar com você -- respondeu a raposa. -- Não sou cativada.

- Ah! Desculpa -- disse o pequeno príncipe.

Mas, depois de refletir, prosseguiu:

- O que significa ``cativar''?

- Você não é daqui -- disse a raposa. -- O que está procurando?

- Estou procurando os homens -- respondeu o pequeno príncipe. -- O que
significa ``cativar''?

- Os homens -- disse a raposa -- têm fuzis e caçam. -- Isso é muito
inconveniente! Eles também criam galinhas. É a única coisa interessante
que fazem. Você está procurando galinhas?

- Não -- disse o pequeno príncipe. -- Estou procurando amigos. O que
significa ``cativar''?

- É algo que já foi esquecido -- disse a raposa. -- Significa ``criar
laços''...

- Criar laços?

- Sim -- respondeu a raposa. -- Para mim, você não passa de um menino
similar a milhares de outros. E eu não preciso de você. E você também
não precisa de mim. Para você, eu não passo de uma raposa similar a
milhares de outras. Mas, se você me prender, iremos precisar um do
outro. Você será pra mim o único do mundo. Eu serei para você a única do
mundo...

- Estou começando a entender -- disse o pequeno príncipe. -- Existe uma
flor... acho que ela me cativou...

- É possível -- disse a raposa. -- Na Terra a gente vê todo tipo de
coisa....

- Oh! Mas não fica na Terra -- disse o pequeno príncipe.

A raposa pareceu intrigada:

- É em um outro planeta?

- Sim.

- Existem caçadores nesse outro planeta?

- Não.

- Isso é interessante. E galinhas?

- Não.

- Pois nada é perfeito -- suspirou a raposa.

Mas a raposa insistiu na sua ideia:

- Minha vida é monótona. Eu caço galinhas e os homens me caçam. Todas as
galinhas se parecem, e todos os homens se parecem. Entedio-me um pouco
com isso. Mas, se você me cativar, minha vida será ensolarada. Eu
conhecerei um ruído de passos que será diferente de todos os outros. Os
outros passos me fazem esconder debaixo da terra. O seu me chamará para
fora da toca, como uma música. E depois imagina! Está vendo ali, no
campo de trigo? Eu não como pão. O trigo pra mim é inútil. As plantações
de trigo não me atraem. E isso é muito triste! Mas você tem cabelos
dourados. Então vai ser maravilhoso quando você me cativar! O trigo, que
é dourado, me fará lembrar de você. E eu amarei o barulho do vento no
trigo...

A raposa se calou e olhou por muito tempo o pequeno príncipe:

- Por favor... me cative! -- pediu ela.

- Eu adoraria -- respondeu o pequeno príncipe --, mas eu não tenho muito
tempo. Tenho amigos para descobrir e muito o que conhecer.

- Só conhecemos o que domesticamos -- disse a raposa. -- Os homens não
têm mais tempo de conhecer nada. Eles compram as coisas prontas no
mercado. Mas, como não existe mercado de amigos, os homens não têm mais
amigos. Se você quer um amigo, me cative!

- O que é preciso fazer? -- perguntou o pequeno príncipe.

- É preciso ter paciência -- respondeu a raposa. -- No início, você vai
se sentar um pouco longe de mim, assim, na grama. Eu vou te olhar com o
canto dos olhos e você não vai me dizer nada. A linguagem é a origem dos
mal-entendidos. Mas, a cada dia, você poderá se aproximar um pouco
mais...

No dia seguinte o pequeno príncipe voltou.

- Seria melhor voltar no mesmo horário -- disse a raposa. -- Se você
vier, por exemplo, às quatro da tarde, começarei a ficar feliz a partir
das três. Quanto mais a hora for avançando, mais feliz eu vou ficar. Às
três, já estarei agitada e inquieta: descobrirei o preço da felicidade!
Mas, se você chega a qualquer momento, nunca vou saber quando devo
preparar meu coração... É preciso criar rituais.

- O que é um ritual? -- perguntou o pequeno príncipe.

- É algo que também foi esquecido -- disse a raposa. -- É o que faz com
que um dia seja diferente dos outros, uma hora, das outras horas. Existe
um ritual, por exemplo, entre meus caçadores. Às quintas-feiras eles
dançam com as moças da vila. Por isso quinta-feira é um dia maravilhoso!
Dou uma volta até a vinha. Se os caçadores dançassem num dia qualquer,
todos os dias seriam parecidos e eu não teria férias.

Assim, o pequeno príncipe cativou a raposa. E quando a hora da partida
se aproximou:

- Ah -- disse a raposa... -- Vou chorar.

- A culpa é sua -- disse o pequeno príncipe --, eu não queria te fazer
sofrer, mas você quis que eu te domesticasse...

- É claro que eu quis -- disse a raposa.

- Mas você vai chorar! -- disse o pequeno príncipe.

- É claro que eu vou -- disse a raposa.

- Mas você não ganha nada com isso!

- É claro que eu ganho -- respondeu a raposa -- por causa da cor do
trigo.

Em seguida, ela acrescentou:

- Vá olhar as rosas. Você vai ver que a sua é única no mundo. Você vai
vir me dizer adeus e eu te presentearei com um segredo.

O pequeno príncipe partiu para ver as rosas:

- Vocês não se parecem em nada com a minha rosa, vocês ainda não são
nada -- ele lhes disse. -- Ninguém cativou vocês e vocês não cativaram
ninguém. São como era a minha raposa. Não passava de uma raposa parecida
a milhares de outras. Mas eu fiz dela uma amiga e agora ela é única no
mundo.

E as rosas ficaram bem incomodadas.

- Vocês são lindas, mas são vazias -- ele ainda lhes disse. -- Não se
pode morrer por vocês. É claro que um homem qualquer que estivesse
passando pensaria que minha rosa se parece com vocês. Mas ela, sozinha,
é mais importante do que vocês todas, porque foi ela que eu reguei. Foi
ela que coloquei na redoma. Foi ela que protegi do vento. Foi por dela
que matei as lagartas (com exceção de duas ou três, para que virassem
borboletas). Porque foi ela que escutei reclamando, se gabando e até
mesmo, às vezes, se calando. Porque ela é a minha rosa.

E voltou-se em direção à raposa:

- Adeus -- disse ele...

- Adeus -- disse a raposa. -- Eis meu segredo. Ele é bem simples: só se
pode ver bem com o coração. O essencial é invisível aos olhos.

- O essencial é invisível aos olhos -- repetiu o pequeno príncipe, para
não se esquecer.

- É o tempo que você perdeu com sua rosa que fez de sua rosa tão
importante.

- É o tempo que eu perdi com minha rosa... -- repetiu o príncipe, para
não se esquecer.

- Os homens esqueceram essa verdade -- disse a raposa. -- Mas você não
pode esquecê-la. Você se torna responsável para sempre por aquilo que
cativa. Você é responsável por sua rosa...

- Eu sou responsável por minha rosa... -- repetiu o pequeno príncipe,
para não se esquecer.

\chapter{XXII}

- Bom dia -- disse o pequeno príncipe.

- Bom dia -- disse o guardador de chaves.

- Eu organizo os viajantes em blocos de mil -- disse o guardador de
chaves. -- Despacho os trens que os transportam, ora para a direita, ora
para a esquerda.

E uma rápida faísca, com um barulho um trovão, fez tremer a cabine de
controle.

- Eles estão muito apressados -- comentou o pequeno príncipe. -- O que
estão procurando?

- Nem o homem da locomotiva sabe -- disse o guardador de chaves.

E trovejou, em sentido inverso, um outra rápida faísca.

- Eles já estão voltando? -- perguntou o pequeno príncipe...

- Não são os mesmos -- disse o guardador de chaves. -- É uma troca.

- Eles não estavam contentes onde estavam?

- Nunca estamos contentes onde estamos. -- respondeu o guardador de
chaves.

E um terceiro lampejo, iluminado, trovejou.

- Eles estão seguindo os primeiros viajantes? -- perguntou o pequeno
príncipe.

- Não estão seguindo nada -- respondeu o controlador. -- Eles dormem lá
dentro, ou ficam a bocejar. Só as crianças amassam a cara contra o
vidro.

- Só as crianças sabem o que estão procurando -- disse o pequeno
príncipe. -- Elas perdem tempo com uma boneca de pano, que passa a ser
muito importante, e se alguém a tira delas, começam a chorar...

- Eles têm sorte -- disse o guardador de chaves.

\chapter{XXIII}

- Bom dia -- disse o pequeno príncipe.

- Bom dia -- respondeu o vendedor.

Era um vendedor de sofisticadas pílulas que matam a sede. Toma-se uma
por semana e não há mais necessidade de tomar água.

- Por que você vende isso? -- perguntou o pequeno príncipe.

- Porque proporciona uma enorme economia de tempo -- respondeu o
vendedor. -- Os especialistas fizeram os cálculos. Economiza-se
cinquenta e três minutos por semana.

- E o que se faz com esses cinquenta e três minutos?

- O que bem entendermos...

``Se eu tivesse cinquenta e três minutos para gastar'' pensou o pequeno
príncipe, ``eu andaria lentamente até uma fonte...''

XXIV

Estávamos no oitavo dia da minha pane no deserto e ao beber a última
gota de minha provisão de água, ouvi a história do comerciante:

- Ah! -- eu disse ao pequeno príncipe -- Suas lembranças são bonitas,
mas eu ainda não consertei meu avião, não tenho mais nada pra beber, e
ficarei muito feliz se eu puder andar lentamente até uma fonte.

- A minha amiga raposa -- disse-me ele...

- Meu rapaz, não mais de trata de raposa!

- Por quê?

- Porque vamos morrer de sede...

Ele não compreendeu meu raciocínio e respondeu:

- É bom ter um amigo, mesmo se vamos morrer. Estou muito feliz por ter
tido uma amiga raposa...

- Ela não mede o perigo -- disse-me. -- Nunca está sentindo nem fome nem
sede. Um pouco de sol já lhe suficiente....

Mas ele me fitou e respondeu ao meu pensamento:

- Eu também estou com sede... vamos procurar um poço...

Fiz um gesto de desprezo: é absurdo procurar um poço, ao acaso, na
imensidade do deserto. Mesmo assim, colocamo-nos a caminho.

Já havíamos andado horas em silêncio quando a noite caiu e as estrelas
começaram a brilhar. Eu as via como num sonho, sentindo um pouco de
febre devido à sede. As palavras do pequeno príncipe dançavam em minha
memória:

- Você também está com sede? -- perguntei-lhe.

Mas ele não respondeu à minha pergunta. Respondeu simplesmente:

- A água também pode ser boa para o coração...

Não compreendi sua resposta, mas permaneci calado... Eu sabia que não
devia interrogá-lo.

Ele estava cansado. Sentou-se. Sentei-me ao seu lado. E, após um
silêncio, ele ainda disse:

- As estrelas são belas por causa de uma flor que não se vê...

Respondi ``é claro'' e, sem nada dizer, fitei as ondas da areia sob a
lua.

- O deserto é lindo -- acrescentou...

Era verdade. Eu sempre gostei do deserto. Sentamo-nos em uma duna de
areia. Não víamos nada. Não ouvíamos nada. E mesmo assim algo brilhou em
silêncio...

- O que deixa o deserto bonito -- disse o pequeno príncipe -- é que em
algum lugar ele esconde um poço...

Fiquei surpreso ao compreender de repente esse misterioso brilho da
areia. Quando menino, eu vivia em uma casa antiga, e a lenda contava que
havia um tesouro escondido ali. É claro que ninguém soube descobri-lo,
nem mesmo, talvez, procurá-lo. Mas ele encantava toda a casa. Minha casa
guardava um segredo no fundo do coração...

- É verdade -- eu disse ao pequeno príncipe --, seja a casa, as estrelas
ou o deserto, o que torna bonito é invisível!

- Fico feliz -- disse ele -- que concorde com minha raposa.

Como o pequeno príncipe adormecia, eu o peguei nos braços e retomei o
caminho. Eu estava emocionado. Ele parecia guardar um frágil tesouro. Eu
tinha a impressão de que não havia nada mais frágil na Terra. Eu olhava,
à luz da lua, sua testa pálida, seus olhos fechados, suas mechas de
cabelo balançando ao vento, e pensava: ``O que estou vendo aqui não
passa de uma casca. A mais importante é invisível...''

Como seus lábios entreabertos esboçavam um meio sorriso, continuei
pensando: ``O que mais me comove nesse pequeno príncipe adormecido é sua
lealdade a uma flor, é a imagem de uma rosa que nele radia como o
lampejo de uma lâmpada, mesmo quando está dormindo...'' E eu o percebi
ainda mais frágil. É preciso proteger bem as lâmpadas: uma ventania é
capaz de apagá-la...

E, caminhando assim, encontrei o poço no dia seguinte ao amanhecer.

\chapter{XXV}

- Os homens -- disse o pequeno príncipe -- enfurnam-se nos trens de alta
velocidade, mas não sabem o que estão procurando. Então eles se agitam e
ficam andando em círculo...

E prosseguiu:

- Isso não vale a pena...

O poço que procurávamos não se parecia com os poços do Saara. Os poços
do Saara não passam de simples buracos cavados na areia. Este se parecia
com um poço de uma vila. Mais ali não havia vila alguma, suspeitei que
estava delirando.

- É estranho -- eu disse ao pequeno príncipe --, tudo está preparado: a
polia, o balde e a corda...

Ele riu, segurou a corda e pôs a polia para funcionar.

A polia gemeu como um velho cata-vento quando o vento tinha dormido por
muito tempo.

- Está ouvindo? -- disse o pequeno príncipe -- O poço canta quando o
despertamos...

Eu não queria que ele fizesse esforço:

- Deixe-me fazer isso -- eu disse --, é muito pesado para você.

Devagar, cuidadosamente, icei o balde até a borda e o instalei ali. O
canto da polia permanecia em meus ouvidos e, na água que continuava
cintilando, via tremer o sol.

- Quero beber esta água -- disse o pequeno príncipe --, me dá um
pouco....

Então entendi o que ele estava procurando!

Levei o balde até seus lábios. Ele bebeu, com os olhos fechados. Era
doce como uma festa. Essa água era mais que um alimento. Tinha nascido
do passo sob as estrelas, do canto da polia, do esforço dos meus braços.
Fazia bem ao coração, como um presente. Quando eu era menino, a luz da
árvore de Natal, a música da missa da meia-noite, a ternura dos
sorrisos, compunham, juntos, toda a iluminação do presente de Natal que
eu ganhava.

- Os homens do teu planeta -- disse o pequeno príncipe -- cultivam cinco
mil rosas num mesmo jardim... e eles não conseguem encontrar o que estão
procurando...

- É, eles não conseguem --respondi.

- E entretanto o que eles estão procurando poderia ser encontrado numa
única rosa ou num gole de água...

- É claro -- respondi.

E o pequeno príncipe prosseguiu:

- Mas os olhos são cegos. É preciso procurar com o coração.

Eu tinha bebido. Respirava com facilidade. A areia, ao nascer do sol,
tem a cor de mel. Eu também estava feliz com essa cor de mel. Porque eu
deveria sentir dó?...

- Você precisa cumprir sua promessa -- disse-me pausadamente o pequeno
príncipe, que voltou a se sentar ao meu lado.

- Que promessa?

- Você sabe... uma focinheira para a minha ovelha... sou responsável por
essa flor!

Eu tirei do bolso meus esboços de desenho. O pequeno príncipe olhou pra
eles e disse, sorrindo:

- Os teus baobás fazem lembrar repolhos...

- Oh!

Logo eu, que tinha tanto orgulho dos baobás!

- E a sua raposa... com essas orelhas que parecem chifres... são
compridas demais!

E ele riu novamente.

- Você está sendo injusto, meu rapaz, eu só sei desenhar jiboias
fechadas e abertas.

- Ah! Não tem problema -- disse ele --, as crianças entendem.

Então rabisquei uma focinheira. E entreguei pra ele com o coração
apertado:

- Você tem planos que eu desconheço...

Ele não respondeu. Mas disse:

- Você se lembra de quando caí na Terra?... Amanhã é aniversário desse
acontecimento...

Depois, após um silêncio, continuou:

- Eu tinha caído aqui por perto...

E enrubesceu.

E, de novo, sem entender o motivo, experimentei uma profunda tristeza.
No entanto, uma pergunta veio-me à cabeça:

- Então não foi por acaso que você andava sozinho naquela manhã em que
te conheci, há oito dias, a milhas e milhas de qualquer região habitada!
Você estava voltando para o ponto da queda?

O pequeno príncipe se enrubesceu ainda mais.

Prossegui, hesitante:

- Era por causa do aniversário?...

O pequeno príncipe enrubesceu outra vez. Ele nunca respondia às
perguntas, mas, quando enrubescemos, isso quer dizer ``sim'', não é?

- Ah -- eu disse --, estou com medo...

Ele respondeu:

- Agora você precisa trabalhar. Tem que voltar para a sua máquina. Eu te
espero aqui. Volte amanhã à noite...

Mas eu não estava tranquilo. Tinha a cabeça na raposa. Corremos o risco
de chorar um pouco quando nos deixamos cativar...

\chapter{XXVI}

Ao lado do poço, havia uma ruína do velho muro de pedra. Ao voltar do
trabalho, na noite seguinte, avistei meu pequeno príncipe sentado no
alto, balançando as pernas. Ouvi-o perguntar:

- Você não se lembra? Não foi bem aqui!

Uma outra voz deve ter respondido algo, porque ele replicou:

- Claro! O dia é o mesmo, já o lugar...

Prossegui meu caminho em direção ao muro. Não via nem escutava ninguém.
No entanto, o pequeno príncipe respondeu novamente:

- Tudo bem. Você vai ver onde começa minha marca na areia. Você só
precisa me esperar. Estarei ali esta noite.

Eu estava a vinte metros da parede e continuava sem enxergar nada.

O pequeno príncipe disse, depois de um silêncio:

- O teu veneno é bom? Tem certeza de que não vai me fazer sofrer por
muito tempo?

Fiz uma pausa, com o coração apertado, mas eu continuava sem entender.

- Agora, pode ir -- disse ele... -- Quero descer!

Então baixei os olhos na direção do pé do muro e saltei! Ao lado do
pequeno príncipe, lá estava uma dessas serpentes amarelas capazes de te
executar em trinta segundos. Enquanto revirava o bolso para pegar meu
revólver, corri, mas, com o barulho que fiz, a serpente se deixou ir
lentamente pela areia, como um fio de água que está se extinguindo, e,
sem muito me apressar, enfiou-se entre as pedras fazendo um leve ruído
de metal.

Cheguei ao muro a tempo para receber nos braços meu pequeno príncipe,
pálido como a neve.

- Que história é essa? Agora você fala com serpentes?

Desfiz o laço de sua eterna echarpe dourada. Molhei suas têmporas e
dei-lhe de beber. Agora eu não ousava pedir-lhe nada mais. Ele me olhou
com seriedade e passou o braço entorno do meu pescoço. Senti seu coração
batendo como o de um quem está morrendo por um tiro de espingarda. Ele
disse:

- Estou feliz que tenha encontrado o que faltava em sua máquina. Agora
vai poder voltar pra casa...

\begin{quote}
- Como você sabe?
\end{quote}

Eu acabava de lhe anunciar que, contra toda e qualquer esperança, eu
tinha encerrado meu trabalho!

Ele não respondeu à minha pergunta, mas acrescentou:

- Eu também volto hoje pra casa...

E depois, meio melancólico:

- É muito mais longe... é muito mais difícil...

Eu sentia que algo de extraordinário estava acontecendo. Eu o apertei em
meus braços como um menininho, e, no entanto, parecia que ele
escorregava em queda vertical num abismo sem que eu pudesse segurá-lo...

Ele tinha o olhar sério, perdido ao longe:

- Eu tenho sua ovelha. E tenho a caixa para a ovelha. E a focinheira...

Ele sorriu com melancolia.

Esperei por muito tempo. Sentia que ele se esquentava pouco a pouco:

- Meu querido, você sentiu medo...

É claro que ele sentiu medo! Mas riu, carinhosamente:

- Sentirei ainda mais medo esta noite...

Gelei novamente com o sentimento do irreparável. Compreendi que me era
insuportável a ideia de nunca mais ouvir aquele riso. Era, para mim,
como uma fonte no deserto.

- Querido, eu ainda quero ouvir seu riso...

- Este noite fará um ano. Minha estrela estará exatamente em cima de
onde caí no ano passado...

- Meu querido, não é um sonho ruim essa história de serpente, encontro e
estrela?

Mas ele não respondeu à minha pergunta. Disse:

- O que é importante, não podemos ver...

- Claro...

- É como a flor. Se você ama uma flor que se encontra em uma estrela, é
agradável, à noite, olhar para o céu. Todas as estrelas estão floridas.

- Claro...

- O mesmo vale para a água. Aquela que você me deu para beber era como
uma música, por causa da polia e da corda... você se lembra?... estava
deliciosa.

- Claro...

- À noite, você olhará as estrelas. É pequeno demais o meu planeta para
que eu te mostre onde fica. Mas é melhor assim. Minha estrela será para
você mais uma das estrelas. Então você vai gostar de olhar para todas as
estrelas... Todas elas serão suas amigas. E, depois, vou te oferecer um
presente...

Ele continuou rindo.

- Ah, meu querido, como eu gosto de ouvir esse riso!

- Pois é justamente esse o meu presente... ele será como a água...

- O que você quer dizer com isso?

- As pessoas têm estrelas que não são as mesmas. Para alguns, os que
viajam, as estrelas são guias. Para outros, elas não passam de luzinhas.
Para outros, os sábios, elas são problemas. Para meu comerciante elas
eram de ouro. Mas todas essas estrelas se calam. Quanto a você, terá
estrelas como ninguém...

- O que você quer dizer?

- Quando você olhar o céu, à noite, porque eu viverei em uma delas,
porque eu estaria em uma delas, rindo, então será pra você como se todas
as estrelas rissem. Você terá estrelas que sabem rir!

E ele continuou rindo.

- E quando se sentir consolado (sempre estamos nos consolando) ficará
feliz por ter me conhecido. Você será para sempre meu amigo. Terá
vontade de rir comigo. E de vez em quando abrirá a janela, assim, só por
prazer... E teus amigos ficarão surpresos em te ver rindo, olhando para
o céu. Então você vai lhes dizer: ``Sim, as estrelas sempre me fazem
rir!'' E eles te acharão louco. Eu terei te pregado uma bela peça...

E ele continuou rindo.

- Será como se eu tivesse te dado, em vez de estrelas, vários guizos que
sabem rir...

E ele riu novamente. Depois ficou sério:

- Esta noite... como você sabe... não venha.

- Não vou te deixar.

- Vai parecer que estou sofrendo... que estou morrendo. É normal. Não
venha ver, não vale a pena...

- Eu não vou te deixar.

Mas ele estava preocupado.

- Se te digo isso... também é por causa da serpente. É preciso impedir
que ela te pique... As serpentes são más. Elas são capazes de picar só
por prazer.

\begin{quote}
- Não vou te deixar.
\end{quote}

Mas algo o tranquilizou:

\begin{quote}
- De qualquer forma, elas não têm veneno para a segunda picada...
\end{quote}

Naquela noite eu não o vi tomar o caminho de casa. Ele fugiu sem fazer
barulho. Quando finalmente o encontrei, ele andava firme, com um passo
rápido. Disse-me apenas:

\begin{quote}
- Ah! Você está aqui...
\end{quote}

E segurou-me pela mão. Mas ele ainda estava atormentado.

- Você se engana. Vai se sentir mal. Eu parecerei morto mas não será
verdade...

Eu me calava.

- Você entende. É longe demais. Eu não posso carregar esse corpo. É
pesado demais.

Eu me calava.

- Será como uma velha casca abandonada. As velhas cascas não são
necessariamente tristes...

Eu me calava.

Ele se desencorajou um pouco. Mas ainda fez um esforço:

- Será gentil, você sabe. Eu também olharei as estrelas. Todas as
estrelas serão de poços com uma roldana enferrujada. Todas as estrelas
me darão de beber...

Eu me calava.

- Será tão divertido! Você terá quinhentos milhões de guizos, eu terei
quinhentos milhões de fontes...

E ele também se calou, porque chorava...

- É aqui. Deixe-me dar um passo sozinho.

E ele se sentou porque tinha medo. Ainda disse:

- Você sabe... a minha flor... sou responsável por ela! E ela é tão
frágil! E tão ingênua. Só tem quatro espinhozinhos para protegê-la desse
mundo...

Eu também me sentei porque não aguentava mais ficar de pé. Ele disse:

- Pronto... Acabou...

Ele hesitou um pouco, depois levantou. Deu um passo. Eu não podia me
mexer.

O máximo que houve foi um clarão dourado perto do seu tornozelo. Por um
momento, ele permaneceu imóvel. Não gritou. Caiu devagar como uma
árvore. Não fez sequer barulho, por causa da areia.

\chapter{XXVII}

Isso já faz seis anos agora... Eu nunca mais contei essa história. Os
camaradas que me viram de novo ficaram muito felizes em me encontrar
vivo. Eu estava triste porém dizia a eles: ``É o cansaço...''.

Hoje em dia estou mais consolado. Quer dizer... não completamente. Mas
tenho a consciência de que ele voltou para seu planeta, quando o dia
amanheceu, não encontrei o seu corpo. Não era um corpo tão pesado
assim... E eu adoro ouvir as estrelas à noite. É como cento e cinquenta
milhões de guizos...

Mas daí aconteceu algo de extraordinário. Esqueci de colocar a correia
de couro na focinheira que havia desenhado para o pequeno príncipe. Ele
nunca pode amarrá-la no carneiro. Por isso me pergunto: ``O que se
passou em seu planeta? Não duvido que o carneiro tenha comido a
flor...''.

Às vezes penso: ``É claro que não! O pequeno príncipe encobre sua flor
todas as noites sob a redoma de vidro e vigia o carneiro...''. Então me
alegro. E todas as estrelas riem gentilmente.

Outras vezes penso: ``Basta distrair-se um pouco! Certa noite, ele
esqueceu sua redoma de vidro, ou o carneiro saiu sem fazer barulho
durante a noite...'' E então os guizos se desmancham em lágrimas!...

Eis um grande mistério. Para vocês que também amam o pequeno príncipe,
assim como para mim, tudo no universo muda se, em algum lugar, não se
sabe onde, um carneiro que não conhecemos talvez tenha comido uma rosa.

Olhe para o céu. Pergunte-se: ``O carneiro comeu ou não a flor?'' E verá
como tudo muda...

E nenhuma pessoa grande jamais compreenderá que isso tenha tanta
importância.

Esta é, para mim, a paisagem mais linda e a mais triste do mundo.
Trata-se da mesma paisagem da página anterior, mas desenhei novamente
para mostrá-la bem. Foi aqui que o pequeno príncipe apareceu na terra,
para depois sumir.

Olhem com atenção essa paisagem para que possam reconhecê-lo se um dia
forem à África, no deserto. Se acontecer de passar por aí, peço-lhes que
não se apressem, aguardem um pouco debaixo da estrela! Caso uma criança
se aproxime, caso ela ria, caso tenha cabelos dourados, caso não
responda quando a interrogamos, com certeza irão adivinhar de quem se
trata. E então sejam gentis! Não me deixem triste: escrevam-me
imediatamente contando que ele voltou...
