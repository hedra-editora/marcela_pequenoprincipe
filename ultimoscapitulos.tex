\section{XXVI}
Il y avait, à côté du puits, une ruine de vieux mur de pierre. Lorsque je
revins de mon travail, le lendemain soir, j'aperçus de loin mon petit prince
assis là"-haut, les jambes pendantes. Et je l'entendis qui parlait:

--- Tu ne t'en souviens donc pas? disait"-il. Ce n'est pas tout à fait ici!

Une autre voix lui répondit sans doute, puisqu'il répliqua:

--- Si! Si! c'est bien le jour, mais ce n'est pas ici l'endroit\ldots{}

Je poursuivis ma marche vers le mur. Je ne voyais ni n'entendais toujours
personne. Pourtant le petit prince répliqua de nouveau:

--- \ldots{} Bien sûr. Tu verras où commence ma trace dans le sable. Tu n'as
qu’à m'y attendre. J'y serai cette nuit.

J'étais à vingt mètres du mur et je ne voyais toujours rien.

Le petit prince dit encore, après un silence:

--- Tu as du bon venin? Tu es sûr de ne pas me faire souffrir longtemps?

Je fis halte, le cœur serré, mais je ne comprenais toujours pas.

--- Maintenant, va"-t'en, dit"-il\ldots{} je veux redescendre!

Alors j'abaissai moi"-même les yeux vers le pied du mur, et je fis un bond!
Il était là, dressé vers le petit prince, un de ces serpents jaunes qui vous
exécutent en trente secondes. Tout en fouillant ma poche pour en tirer mon
revolver, je pris le pas de course, mais, au bruit que je fis, le serpent se
laissa dou cement couler dans le sable, comme un jet d'eau qui meurt, et,
sans trop se presser, se faufila entre les pierres avec un léger bruit de
métal.

Je parvins au mur juste à temps pour y recevoir dans les bras mon petit
bonhomme de prince, pâle comme la neige.

--- Quelle est cette histoire"-là! Tu parles maintenant avec les serpents!

J'avais défait son éternel cache"-nez d'or. Je lui avais mouillé les tempes
et l'avais fait boire. Et maintenant je n'osais plus rien lui demander. Il
me regarda gravement et m'entoura le cou de ses bras. Je sentais battre son
cœur comme celui d'un oiseau qui meurt, quand on l'atiré à la carabine. Il
me dit:

--- Je suis content que tu aies trouvé ce qui manquait à ta machine. Tu vas
pouvoir rentrer chez toi...

Comment sais"-tu!

Je venais justement lui annoncer que, contre toute espérance, j'avais réussi
mon travail!

Il ne répondit rien à ma question, mais il ajouta:

--- Moi aussi, aujourd'hui, je rentre chez moi\ldots{}

Puis, mélancolique:

--- C'est bien plus loin\ldots{} c'est bien plus difficile\ldots{}

Je sentais bien qu'il se passait quelque chose d'extraordinaire. Je le serrais
dans les bras comme un petit enfant, et cependant il me semblait qu'il coulait
verticalement dans un abîme sans que je puisse rien pour le retenir\ldots{}

Il avait le regard sérieux, perdu très loin:

--- J'ai ton mouton. Et j'ai la caisse pour le mouton. Et j'ai la muselière\ldots{}

Et il sourit avec mélancolie.

J'attendis longtemps. Je sentais qu'il se réchauffait peu à peu:

--- Petit bonhomme, tu as eu peur\ldots{}

Il avait eu peur, bien sûr! Mais il rit doucement:

--- J'aurai bien plus peur ce soir\ldots{}

De nouveau je me sentis glacé par le sentiment de l'irréparable. Et je compris
que je ne supportais pas l'idée de ne plus jamais entendre ce rire. C'était
pour moi comme une fontaine dans le désert.

--- Petit bonhomme, je veux encore t'entendre rire\ldots{}

Mais il me dit:

--- Cette nuit, ça fera un an. Mon étoile se trouvera juste au"-dessus de
l'endroit où je suis tombé l'année dernière\ldots{}

91

\section{XXVII}

Et maintenant bien sûr, ça fait six ans déjà\ldots{} Je n'ai jamais encore
raconté cette histoire. Les camarades qui m'ont revu ont été bien contents
de me revoir vivant. J'étais triste mais je leur disais ``C'est la fatigue\ldots{}''

Maintenant je me suis un peu consolé. C’est"-à"-dire\ldots{} pas tout à fait.
Mais je sais bien qu'il est revenu à sa planète, car, au lever du jour, je
n'ai pas retrouvé son corps. Ce n'était pas un corps tellement lourd\ldots{}
Et j'aime la nuit écouter les étoiles. C'est comme cinq cents millions de
grelots\ldots{}

Mais voilà qu'il se passe quelque chose d'extraordinaire. La muselière que 
j'ai dessinée pour le petit prince, j'ai oublié d'y ajouter la courroie de
cuir! Il n'aura jamais pu l'attacher au mouton. Alors je me demande: ``Que
s'est"-il passé sur sa planète? Peut"-être bien que le mouton a mangé la
fleur\ldots{}''

Tantôt je me dis: ``Sûrement non! Le petit prince enferme sa fleur toutes les
nuits sous son globe de verre, et il surveille bien son mouton\ldots{}'' Alors
je suis heureux. Et toutes les étoiles rient doucement.

Tantôt je me dis: ``On est distrait une fois ou l'autre, et ça suffit! Il a
oublié, un soir, le globe de verre, ou bien le mouton est sorti sans bruit
pendant la nuit\ldots{}'' Alors les grelots se changent tous en larmes!\ldots{}

\medskip

C'est là un bien grand mystère. Pour vous qui aimez aussi le petit prince,
comme pour moi, rien de l'univers n'est semblable si quelque part, on ne sait
où, un mouton que nous ne connaissons pas a, oui ou non, mangé une rose\ldots{}

Regardez le ciel. Demandez"-vous: ``Le mouton oui ou non a"-t"-il mangé la
fleur?'' Et vous verrez comme tout change\ldots{}

Et aucune grande personne ne comprendra jamais que ça a tellement d'importance!

Ça c'est, pour moi, le plus beau et le plus triste paysage du monde. C’est
le même paysage que celui de la page précédente, mais je l'ai dessiné une fois
encore pour bien vous le montrer. C'est ici que le petit prince a apparu sur
terre, puis disparu.

Regardez attentivement ce paysage afin d'être sûrs de le reconnaître, si vous
voyagez un jour en Afrique, dans le désert. Et, s'il vous arrive de passer par
là, je vous en supplie, ne vous pressez pas, attendez un peu juste sous l'étoile!
Si alors un enfant vient a vous, s'il rit, s'il a des cheveux d'or, s'il ne
répond pas quand on l'interroge, vous devinerez bien qui il est. Alors soyez
gentils! Ne me laissez pas tellement triste: écrivez"-moi vite qu'il est
revenu\ldots{}
