\section{XXVI}
Il y avait, à côté du puits, une ruine de vieux mur de pierre. Lorsque je
revins de mon travail, le lendemain soir, j'aperçus de loin mon petit prince
assis là"-haut, les jambes pendantes. Et je l'entendis qui parlait:

--- Tu ne t'en souviens donc pas? disait"-il. Ce n'est pas tout à fait ici!

Une autre voix lui répondit sans doute, puisqu'il répliqua:

--- Si! Si! c'est bien le jour, mais ce n'est pas ici l'endroit\ldots{}

Je poursuivis ma marche vers le mur. Je ne voyais ni n'entendais toujours
personne. Pourtant le petit prince répliqua de nouveau:

--- \ldots{} Bien sûr. Tu verras où commence ma trace dans le sable. Tu n'as
qu’à m'y attendre. J'y serai cette nuit.

J'étais à vingt mètres du mur et je ne voyais toujours rien.

Le petit prince dit encore, après un silence:

--- Tu as du bon venin? Tu es sûr de ne pas me faire souffrir longtemps?

Je fis halte, le cœur serré, mais je ne comprenais toujours pas.

--- Maintenant, va"-t'en, dit"-il\ldots{} je veux redescendre!

Alors j'abaissai moi"-même les yeux vers le pied du mur, et je fis un bond!
Il était là, dressé vers le petit prince, un de ces serpents jaunes qui vous
exécutent en trente secondes. Tout en fouillant ma poche pour en tirer mon
revolver, je pris le pas de course, mais, au bruit que je fis, le serpent se
laissa dou cement couler dans le sable, comme un jet d'eau qui meurt, et,
sans trop se presser, se faufila entre les pierres avec un léger bruit de
métal.

Je parvins au mur juste à temps pour y recevoir dans les bras mon petit
bonhomme de prince, pâle comme la neige.

--- Quelle est cette histoire"-là! Tu parles maintenant avec les serpents!

J'avais défait son éternel cache"-nez d'or. Je lui avais mouillé les tempes
et l'avais fait boire. Et maintenant je n'osais plus rien lui demander. Il
me regarda gravement et m'entoura le cou de ses bras. Je sentais battre son
cœur comme celui d'un oiseau qui meurt, quand on l'atiré à la carabine. Il
me dit:

--- Je suis content que tu aies trouvé ce qui manquait à ta machine. Tu vas
pouvoir rentrer chez toi...

Comment sais"-tu!

Je venais justement lui annoncer que, contre toute espérance, j'avais réussi
mon travail!

Il ne répondit rien à ma question, mais il ajouta:

--- Moi aussi, aujourd'hui, je rentre chez moi\ldots{}

Puis, mélancolique:

--- C'est bien plus loin\ldots{} c'est bien plus difficile\ldots{}

Je sentais bien qu'il se passait quelque chose d'extraordinaire. Je le serrais
dans les bras comme un petit enfant, et cependant il me semblait qu'il coulait
verticalement dans un abîme sans que je puisse rien pour le retenir\ldots{}

Il avait le regard sérieux, perdu très loin:

--- J'ai ton mouton. Et j'ai la caisse pour le mouton. Et j'ai la muselière\ldots{}

Et il sourit avec mélancolie.

J'attendis longtemps. Je sentais qu'il se réchauffait peu à peu:

--- Petit bonhomme, tu as eu peur\ldots{}

Il avait eu peur, bien sûr! Mais il rit doucement:

--- J'aurai bien plus peur ce soir\ldots{}

De nouveau je me sentis glacé par le sentiment de l'irréparable. Et je compris
que je ne supportais pas l'idée de ne plus jamais entendre ce rire. C'était
pour moi comme une fontaine dans le désert.

--- Petit bonhomme, je veux encore t'entendre rire\ldots{}

Mais il me dit:

--- Cette nuit, ça fera un an. Mon étoile se trouvera juste au"-dessus de
l'endroit où je suis tombé l'année dernière\ldots{}

--- Petit bonhomme, n'est-ce pas que c'est un mauvais rêve cette histoire de
serpent et de rendez"-vous et d'étoile\ldots{}

Mais il ne répondit pas à ma question. Il me dit:

--- Ce qui est important, ça ne se voit pas\ldots{}

--- Bien sûr\ldots{}

--- C'est comme pour la fleur. Si tu aimes une fleur qui se trouve dans une
étoile, c'est doux, la nuit, de regarder le ciel. Toutes les étoiles sont
fleuries.

--- Bien sûr\ldots{}

--- C'est comme pour l'eau. Celle que tu m'as donnée à boire était comme une
musique, à cause de la poulie et de la corde\ldots{} tu te rappelles\ldots{}
elle était bonne.

--- Bien sûr\ldots{}

--- Tu regarderas, la nuit, les étoiles. C'est trop petit chez moi pour que
je te montre où se trouve la mienne. C'est mieux comme ça. Mon étoile, ça
sera pour toi une des étoiles. Alors, toutes les étoiles, tu aimeras les
regarder\ldots{} Elles seront toutes tes amies. Et puis je vais te faire un
cadeau\ldots{}

ll rit encore.

--- Ah! petit bonhomme, petit bonhomme, j'aime entendre ce rire!

--- Justement ce sera mon cadeau\ldots{} ce sera comme pour l'eau\ldots{}

--- Que veux"-tu dire?

--- Les gens ont des étoiles qui ne sont pas les mêmes. Pour les uns, qui
voyagent, les étoiles sont des guides. Pour d'autres elles ne sont rien que
de petites lumières. Pour d’autres, qui sont savants, elles sont des problèmes.
Pour mon businessman elles étaient de l'or. Mais toutes ces étoiles"-là se
taisent. Toi, tu auras des étoiles comme personne n'en a\ldots{}

--- Que veux"-tu dire?

--- Quand tu regarderas le ciel, la nuit, puisque j'habiterai dans l'une
d'elles, puisque je rirai dans l'une d'elles, alors ce sera pour toi comme si
riaient toutes les étoiles. Tu auras, toi, des étoiles qui savent rire!

Et il rit encore.

--- Et quand tu seras consolé (on se console toujours) tu seras content de
m'avoir connu. Tu seras toujours mon ami. Tu auras envie de rire avec moi. Et
tu ouvriras parfois ta fenêtre, comme ça, pour le plaisir\ldots{} Et tes amis
seront bien étonnés de te voir rire en regardant le ciel. Alors tu leur diras:
``Oui, les étoiles, ça me fait toujours rire!'' Et ils te croiront fou. Je
t'aurai joué un bien vilain tour\ldots{}

Et il rit encore.

--- Ce sera comme si je t'avais donné, au lieu d'étoiles, des tas de petits
grelots qui savent rire\ldots{}

Et il rit encore. Puis il redevint sérieux:

--- Cette nuit\ldots{} tu sais\ldots{} ne viens pas.

--- Je ne te quitterai pas.

--- J'aurai l'air d'avoir mal\ldots{} j'aurai un peu l'air de mourir. C'est
comme ça. Ne viens pas voir ça, ce n'est pas la peine\ldots{}

--- Je ne te quitterai pas.

Mais il était soucieux.

--- Je te dis ça\ldots{} c’est à cause aussi du serpent. Il ne l'aut pas qu'il
te morde\ldots{} Les serpents, c'est méchant. Ça peut mordre pour le plaisir\ldots{}

--- Je ne te quitterai pas.

Mais quelque chose le rassura:

--- C'est vrai qu'ils n'ont plus de venin pour la seconde morsure\ldots{}

\medskip

Cette nuit"-là je ne le vis pas se mettre en route. Il s’était évadé sans bruit.
Quand je réussis à le rejoindre il marchait décidé, d’un pas rapide. Il me dit
seulement:

--- Ah! tu es là\ldots{}

Et il me prit par la main. Mais il se tourmenta encore:

--- Tu as eu tort. Tu auras de la peine. J'aurai l'air d'être mort et ce ne
sera pas vrai\ldots{}

Moi je me taisais.

--- Tu comprends. C'est trop loin. Je ne peux pas emporter ce corps"-là. C'est
trop lourd.

Moi je me taisais.

--- Mais ce sera comme une vieille écorce abandonnée. Ce n'est pas triste les
vieilles écorces\ldots{} I

Moi je me taisais.

Il se découragea un peu. Mais il fit encore un effort:

--- Ce sera gentil, tu sais. Moi aussi, je regarderai les étoiles. Toutes les
étoiles seront des puits avec une poulie rouillée. Toutes les étoiles me
verseront à boire\ldots{}

Moi je me taisais.

--- Ce sera tellement amusant! Tu auras cinq cents millions de grelots, j'aurai
cinq cents millions de fontaines\ldots{}

Et il se tu t au ssi, parce qu’ il pleu rait...

\medskip

--- C'est là. Laisse"-moi faire un pas tout seul.

Et il s'assit parce qu'il avait peur. Il dit encore:

--- Tu sais\ldots{} ma fleur\ldots{} J'en suis responsable! Et elle est tellement
faible! Et elle est tellement naïve. Elle a quatre épines de rien du tout pour
la protéger contre le monde\ldots{}

Moi je m'assis parce que je ne pouvais plus me tenir debout. Il dit:

--- Voilà\ldots{} C'est tout\ldots{}

Il hésita encore un peu, puis il se releva. Il fit un pas. Moi je ne pouvais pas
bouger.

Il n'y eut rien qu’un éclair jaune près de sa cheville. Il demeura un instant
immobile. Il ne cria pas. Il tomba doucement comme tombe un arbre. Ça ne fit même
pas de bruit, à cause du sable.

\section{XXVII}

Et maintenant bien sûr, ça fait six ans déjà\ldots{} Je n'ai jamais encore
raconté cette histoire. Les camarades qui m'ont revu ont été bien contents
de me revoir vivant. J'étais triste mais je leur disais ``C'est la fatigue\ldots{}''

Maintenant je me suis un peu consolé. C’est"-à"-dire\ldots{} pas tout à fait.
Mais je sais bien qu'il est revenu à sa planète, car, au lever du jour, je
n'ai pas retrouvé son corps. Ce n'était pas un corps tellement lourd\ldots{}
Et j'aime la nuit écouter les étoiles. C'est comme cinq cents millions de
grelots\ldots{}

Mais voilà qu'il se passe quelque chose d'extraordinaire. La muselière que 
j'ai dessinée pour le petit prince, j'ai oublié d'y ajouter la courroie de
cuir! Il n'aura jamais pu l'attacher au mouton. Alors je me demande: ``Que
s'est"-il passé sur sa planète? Peut"-être bien que le mouton a mangé la
fleur\ldots{}''

Tantôt je me dis: ``Sûrement non! Le petit prince enferme sa fleur toutes les
nuits sous son globe de verre, et il surveille bien son mouton\ldots{}'' Alors
je suis heureux. Et toutes les étoiles rient doucement.

Tantôt je me dis: ``On est distrait une fois ou l'autre, et ça suffit! Il a
oublié, un soir, le globe de verre, ou bien le mouton est sorti sans bruit
pendant la nuit\ldots{}'' Alors les grelots se changent tous en larmes!\ldots{}

\medskip

C'est là un bien grand mystère. Pour vous qui aimez aussi le petit prince,
comme pour moi, rien de l'univers n'est semblable si quelque part, on ne sait
où, un mouton que nous ne connaissons pas a, oui ou non, mangé une rose\ldots{}

Regardez le ciel. Demandez"-vous: ``Le mouton oui ou non a"-t"-il mangé la
fleur?'' Et vous verrez comme tout change\ldots{}

Et aucune grande personne ne comprendra jamais que ça a tellement d'importance!

Ça c'est, pour moi, le plus beau et le plus triste paysage du monde. C’est
le même paysage que celui de la page précédente, mais je l'ai dessiné une fois
encore pour bien vous le montrer. C'est ici que le petit prince a apparu sur
terre, puis disparu.

Regardez attentivement ce paysage afin d'être sûrs de le reconnaître, si vous
voyagez un jour en Afrique, dans le désert. Et, s'il vous arrive de passer par
là, je vous en supplie, ne vous pressez pas, attendez un peu juste sous l'étoile!
Si alors un enfant vient a vous, s'il rit, s'il a des cheveux d'or, s'il ne
répond pas quand on l'interroge, vous devinerez bien qui il est. Alors soyez
gentils! Ne me laissez pas tellement triste: écrivez"-moi vite qu'il est
revenu\ldots{}
